% generated by GAPDoc2LaTeX from XML source (Frank Luebeck)
\documentclass[a4paper,11pt]{report}

\usepackage[top=37mm,bottom=37mm,left=27mm,right=27mm]{geometry}
\sloppy
\pagestyle{myheadings}
\usepackage{amssymb}
\usepackage[utf8]{inputenc}
\usepackage{makeidx}
\makeindex
\usepackage{color}
\definecolor{FireBrick}{rgb}{0.5812,0.0074,0.0083}
\definecolor{RoyalBlue}{rgb}{0.0236,0.0894,0.6179}
\definecolor{RoyalGreen}{rgb}{0.0236,0.6179,0.0894}
\definecolor{RoyalRed}{rgb}{0.6179,0.0236,0.0894}
\definecolor{LightBlue}{rgb}{0.8544,0.9511,1.0000}
\definecolor{Black}{rgb}{0.0,0.0,0.0}

\definecolor{linkColor}{rgb}{0.0,0.0,0.554}
\definecolor{citeColor}{rgb}{0.0,0.0,0.554}
\definecolor{fileColor}{rgb}{0.0,0.0,0.554}
\definecolor{urlColor}{rgb}{0.0,0.0,0.554}
\definecolor{promptColor}{rgb}{0.0,0.0,0.589}
\definecolor{brkpromptColor}{rgb}{0.589,0.0,0.0}
\definecolor{gapinputColor}{rgb}{0.589,0.0,0.0}
\definecolor{gapoutputColor}{rgb}{0.0,0.0,0.0}

%%  for a long time these were red and blue by default,
%%  now black, but keep variables to overwrite
\definecolor{FuncColor}{rgb}{0.0,0.0,0.0}
%% strange name because of pdflatex bug:
\definecolor{Chapter }{rgb}{0.0,0.0,0.0}
\definecolor{DarkOlive}{rgb}{0.1047,0.2412,0.0064}


\usepackage{fancyvrb}

\usepackage{mathptmx,helvet}
\usepackage[T1]{fontenc}
\usepackage{textcomp}


\usepackage[
            pdftex=true,
            bookmarks=true,        
            a4paper=true,
            pdftitle={Written with GAPDoc},
            pdfcreator={LaTeX with hyperref package / GAPDoc},
            colorlinks=true,
            backref=page,
            breaklinks=true,
            linkcolor=linkColor,
            citecolor=citeColor,
            filecolor=fileColor,
            urlcolor=urlColor,
            pdfpagemode={UseNone}, 
           ]{hyperref}

\newcommand{\maintitlesize}{\fontsize{50}{55}\selectfont}

% write page numbers to a .pnr log file for online help
\newwrite\pagenrlog
\immediate\openout\pagenrlog =\jobname.pnr
\immediate\write\pagenrlog{PAGENRS := [}
\newcommand{\logpage}[1]{\protect\write\pagenrlog{#1, \thepage,}}
%% were never documented, give conflicts with some additional packages

\newcommand{\GAP}{\textsf{GAP}}

%% nicer description environments, allows long labels
\usepackage{enumitem}
\setdescription{style=nextline}

%% depth of toc
\setcounter{tocdepth}{1}





%% command for ColorPrompt style examples
\newcommand{\gapprompt}[1]{\color{promptColor}{\bfseries #1}}
\newcommand{\gapbrkprompt}[1]{\color{brkpromptColor}{\bfseries #1}}
\newcommand{\gapinput}[1]{\color{gapinputColor}{#1}}


\begin{document}

\logpage{[ 0, 0, 0 ]}
\begin{titlepage}
\mbox{}\vfill

\begin{center}{\maintitlesize \textbf{ ToricVarieties \mbox{}}}\\
\vfill

\hypersetup{pdftitle= ToricVarieties }
\markright{\scriptsize \mbox{}\hfill  ToricVarieties  \hfill\mbox{}}
{\Huge \textbf{ A package to handle toric varieties \mbox{}}}\\
\vfill

{\Huge  2022.07.13 \mbox{}}\\[1cm]
{ 13 July 2022 \mbox{}}\\[1cm]
\mbox{}\\[2cm]
{\Large \textbf{ Sebastian Gutsche\\
    \mbox{}}}\\
{\Large \textbf{ Martin Bies\\
    \mbox{}}}\\
\hypersetup{pdfauthor= Sebastian Gutsche\\
    ;  Martin Bies\\
    }
\end{center}\vfill

\mbox{}\\
{\mbox{}\\
\small \noindent \textbf{ Sebastian Gutsche\\
    }  Email: \href{mailto://gutsche@mathematik.uni-siegen.de} {\texttt{gutsche@mathematik.uni-siegen.de}}\\
  Homepage: \href{https://sebasguts.github.io} {\texttt{https://sebasguts.github.io}}\\
  Address: \begin{minipage}[t]{8cm}\noindent
 Department Mathematik\\
 Universit{\"a}t Siegen\\
 Walter-Flex-Stra{\ss}e 3\\
 57072 Siegen\\
 Germany\\
 \end{minipage}
}\\
{\mbox{}\\
\small \noindent \textbf{ Martin Bies\\
    }  Email: \href{mailto://martin.bies@alumni.uni-heidelberg.de} {\texttt{martin.bies@alumni.uni-heidelberg.de}}\\
  Homepage: \href{https://martinbies.github.io/} {\texttt{https://martinbies.github.io/}}\\
  Address: \begin{minipage}[t]{8cm}\noindent
 Department of Mathematics \\
 University of Pennsylvania \\
 David Rittenhouse Laboratory \\
 209 S 33rd St \\
 Philadelphia \\
 PA 19104\\
 \end{minipage}
}\\
\end{titlepage}

\newpage\setcounter{page}{2}
{\small 
\section*{Copyright}
\logpage{[ 0, 0, 1 ]}
 This package may be distributed under the terms and conditions of the GNU
Public License Version 2 or (at your option) any later version. \mbox{}}\\[1cm]
\newpage

\def\contentsname{Contents\logpage{[ 0, 0, 2 ]}}

\tableofcontents
\newpage

     
\chapter{\textcolor{Chapter }{Introduction}}\label{Chapter_Introduction}
\logpage{[ 1, 0, 0 ]}
\hyperdef{L}{X7DFB63A97E67C0A1}{}
{
  

 
\section{\textcolor{Chapter }{What is the goal of the ToricVarieties package?}}\label{Chapter_Introduction_Section_What_is_the_goal_of_the_ToricVarieties_package}
\logpage{[ 1, 1, 0 ]}
\hyperdef{L}{X7F1616D17C9AA831}{}
{
  

 \emph{ToricVarieties} provides data structures to handle toric varieties by their commutative
algebra structure and by their combinatorics. For combinatorics, it uses the \emph{Convex} package. Its goal is to provide a suitable framework to work with toric
varieties. All combinatorial structures mentioned in this manual are the ones
from \emph{Convex}. 

 }

 }

   
\chapter{\textcolor{Chapter }{Installation of the ToricVarieties Package}}\label{Chapter_Installation_of_the_ToricVarieties_Package}
\logpage{[ 2, 0, 0 ]}
\hyperdef{L}{X80C6D4EB7BE8CEF0}{}
{
  

 
\begin{itemize}
\item  To install this package just extract the package's archive file to the GAP pkg
directory. 
\item  By default the \emph{ToricVarieties} package is not automatically loaded by \emph{GAP} when it is installed. You must load the package with the following command,
before its functions become avaiable: \emph{LoadPackage( "ToricVarieties" );} 
\item  Please, send me an e-mail if you have any questions, remarks, suggestions,
etc. concerning this package. Also, I would be please to hear about
applications of this package and about any suggestions for new methods to add
to the package. 
\end{itemize}
 

 Sebastian Gutsche 

 }

   
\chapter{\textcolor{Chapter }{Toric Varieties}}\label{Chapter_Toric_Varieties}
\logpage{[ 3, 0, 0 ]}
\hyperdef{L}{X866558FA7BC3F2C8}{}
{
  

 
\section{\textcolor{Chapter }{Toric Varieties: Examples}}\label{Chapter_Toric_Varieties_Section_Toric_Varieties_Examples}
\logpage{[ 3, 1, 0 ]}
\hyperdef{L}{X802337377FDC8121}{}
{
  

 
\subsection{\textcolor{Chapter }{The Hirzebruch surface of index 5}}\label{Chapter_Toric_Varieties_Section_Toric_Varieties_Examples_Subsection_The_Hirzebruch_surface_of_index_5}
\logpage{[ 3, 1, 1 ]}
\hyperdef{L}{X7F674AD387A33155}{}
{
  
\begin{Verbatim}[commandchars=!JK,fontsize=\small,frame=single,label=Example]
  !gappromptJgap>K !gapinputJH5 := Fan( [[-1,5],[0,1],[1,0],[0,-1]],[[1,2],[2,3],[3,4],[4,1]] );K
  <A fan in |R^2>
  !gappromptJgap>K !gapinputJH5 := ToricVariety( H5 );K
  <A toric variety of dimension 2>
  !gappromptJgap>K !gapinputJIsComplete( H5 );K
  true
  !gappromptJgap>K !gapinputJIsSimplicial( H5 );K
  true
  !gappromptJgap>K !gapinputJIsAffine( H5 );K
  false
  !gappromptJgap>K !gapinputJIsOrbifold( H5 );K
  true
  !gappromptJgap>K !gapinputJIsProjective( H5 );K
  true
  !gappromptJgap>K !gapinputJithBettiNumber( H5, 0 );K
  1
  !gappromptJgap>K !gapinputJDimensionOfTorusfactor( H5 );K
  0
  !gappromptJgap>K !gapinputJLength( AffineOpenCovering( H5 ) );K
  4
  !gappromptJgap>K !gapinputJMorphismFromCoxVariety( H5 );K
  <A "homomorphism" of right objects>
  !gappromptJgap>K !gapinputJCartierTorusInvariantDivisorGroup( H5 );K
  <A free left submodule given by 8 generators>
  !gappromptJgap>K !gapinputJTorusInvariantPrimeDivisors( H5 );K
  [ <A prime divisor of a toric variety with coordinates ( 1, 0, 0, 0 )>,
    <A prime divisor of a toric variety with coordinates ( 0, 1, 0, 0 )>,
    <A prime divisor of a toric variety with coordinates ( 0, 0, 1, 0 )>,
    <A prime divisor of a toric variety with coordinates ( 0, 0, 0, 1 )> ]
  !gappromptJgap>K !gapinputJP := TorusInvariantPrimeDivisors( H5 );K
  [ <A prime divisor of a toric variety with coordinates ( 1, 0, 0, 0 )>,
    <A prime divisor of a toric variety with coordinates ( 0, 1, 0, 0 )>,
    <A prime divisor of a toric variety with coordinates ( 0, 0, 1, 0 )>,
    <A prime divisor of a toric variety with coordinates ( 0, 0, 0, 1 )> ]
  !gappromptJgap>K !gapinputJA := P[ 1 ] - P[ 2 ] + 4*P[ 3 ];K
  <A divisor of a toric variety with coordinates ( 1, -1, 4, 0 )>
  !gappromptJgap>K !gapinputJA;K
  <A divisor of a toric variety with coordinates ( 1, -1, 4, 0 )>
  !gappromptJgap>K !gapinputJIsAmple( A );K
  false
  !gappromptJgap>K !gapinputJWeilDivisorsOfVariety( H5 );;K
  !gappromptJgap>K !gapinputJCoordinateRingOfTorus( H5 );K
  Q[x1,x1_,x2,x2_]/( x1*x1_-1, x2*x2_-1 )
  !gappromptJgap>K !gapinputJCoordinateRingOfTorus( H5,"x" );K
  Q[x1,x1_,x2,x2_]/( x1*x1_-1, x2*x2_-1 )
  !gappromptJgap>K !gapinputJD:=CreateDivisor( [ 0,0,0,0 ],H5 );K
  <A divisor of a toric variety with coordinates 0>
  !gappromptJgap>K !gapinputJBasisOfGlobalSections( D );K
  [ |[ 1 ]| ]
  !gappromptJgap>K !gapinputJD:=Sum( P );K
  <A divisor of a toric variety with coordinates ( 1, 1, 1, 1 )>
  !gappromptJgap>K !gapinputJBasisOfGlobalSections(D);K
  [ |[ x1_ ]|, |[ x1_*x2 ]|, |[ 1 ]|, |[ x2 ]|,
    |[ x1 ]|, |[ x1*x2 ]|, |[ x1^2*x2 ]|, 
    |[ x1^3*x2 ]|, |[ x1^4*x2 ]|, |[ x1^5*x2 ]|, 
    |[ x1^6*x2 ]| ]
  !gappromptJgap>K !gapinputJdivi := DivisorOfCharacter( [ 1,2 ],H5 );K
  <A principal divisor of a toric variety with coordinates ( 9, -2, 2, 1 )>
  !gappromptJgap>K !gapinputJBasisOfGlobalSections( divi );K
  [ |[ x1_*x2_^2 ]| ]
  !gappromptJgap>K !gapinputJZariskiCotangentSheafViaPoincareResidueMap( H5 );;K
  !gappromptJgap>K !gapinputJZariskiCotangentSheafViaEulerSequence( H5 );;K
  !gappromptJgap>K !gapinputJEQ( H5, ProjectiveSpace( 2 ) );K
  false
  !gappromptJgap>K !gapinputJH5B1 := BlowUpOnIthMinimalTorusOrbit( H5, 1 );K
  <A toric variety of dimension 2>
  #@if IsPackageMarkedForLoading( "TopcomInterface", ">= 2021.08.12" )
  !gappromptJgap>K !gapinputJH5_version2 := DeriveToricVarietiesFromGrading( [[0,1,1,0],[1,0,-5,1]], false );K
  [ <A toric variety of dimension 2> ]
  !gappromptJgap>K !gapinputJH5_version3 := ToricVarietyFromGrading( [[0,1,1,0],[1,0,-5,1]] );K
  <A toric variety of dimension 2>
  #@fi
  !gappromptJgap>K !gapinputJNameOfVariety( H5 );K
  "H_5"
  !gappromptJgap>K !gapinputJDisplay( H5 );K
  A projective normal toric variety of dimension 2.
  The torus of the variety is RingWithOne( ... ).
  The class group is <object> and the Cox ring is RingWithOne( ... ).
\end{Verbatim}
 Another example 
\begin{Verbatim}[commandchars=!@B,fontsize=\small,frame=single,label=Example]
  !gapprompt@gap>B !gapinput@P2 := ProjectiveSpace( 2 );B
  <A projective toric variety of dimension 2>
  !gapprompt@gap>B !gapinput@IsNormalVariety( P2 );B
  true
  !gapprompt@gap>B !gapinput@AffineCone( P2 );B
  <An affine normal toric variety of dimension 3>
  !gapprompt@gap>B !gapinput@PolytopeOfVariety( P2 );B
  <A polytope in |R^2 with 3 vertices>
  !gapprompt@gap>B !gapinput@IsIsomorphicToProjectiveSpace( P2 );B
  true
  !gapprompt@gap>B !gapinput@IsIsomorphicToProjectiveSpace( H5 );B
  false
  !gapprompt@gap>B !gapinput@Length( MonomsOfCoxRingOfDegree( P2, [1,2,3] ) );B
  28
  !gapprompt@gap>B !gapinput@IsDirectProductOfPNs( P2 * P2 );B
  true
  !gapprompt@gap>B !gapinput@IsDirectProductOfPNs( P2 * H5 );B
  false
\end{Verbatim}
 

 }

 
\subsection{\textcolor{Chapter }{A smooth, complete toric variety which is not projective}}\label{Chapter_Toric_Varieties_Section_Toric_Varieties_Examples_Subsection_A_smooth_complete_toric_variety_which_is_not_projective}
\logpage{[ 3, 1, 2 ]}
\hyperdef{L}{X8524CDF9849408FF}{}
{
  
\begin{Verbatim}[commandchars=!@B,fontsize=\small,frame=single,label=Example]
  !gapprompt@gap>B !gapinput@rays := [ [1,0,0], [-1,0,0], [0,1,0], [0,-1,0], [0,0,1], [0,0,-1],B
  !gapprompt@>B !gapinput@          [2,1,1], [1,2,1], [1,1,2], [1,1,1] ];B
  [ [ 1, 0, 0 ], [ -1, 0, 0 ], [ 0, 1, 0 ], [ 0, -1, 0 ], [ 0, 0, 1 ], [ 0, 0, -1 ], 
  [ 2, 1, 1 ], [ 1, 2, 1 ], [ 1, 1, 2 ], [ 1, 1, 1 ] ]
  !gapprompt@gap>B !gapinput@cones := [ [1,3,6], [1,4,6], [1,4,5], [2,3,6], [2,4,6], [2,3,5], [2,4,5],B
  !gapprompt@>B !gapinput@           [1,5,9], [3,5,8], [1,3,7], [1,7,9], [5,8,9], [3,7,8],B
  !gapprompt@>B !gapinput@           [7,9,10], [8,9,10], [7,8,10] ];B
  [ [ 1, 3, 6 ], [ 1, 4, 6 ], [ 1, 4, 5 ], [ 2, 3, 6 ], [ 2, 4, 6 ], [ 2, 3, 5 ],
    [ 2, 4, 5 ], [ 1, 5, 9 ], [ 3, 5, 8 ], [ 1, 3, 7 ], [ 1, 7, 9 ], [ 5, 8, 9 ], 
    [ 3, 7, 8 ], [ 7, 9, 10 ], [ 8, 9, 10 ], [ 7, 8, 10 ] ]
  !gapprompt@gap>B !gapinput@F := Fan( rays, cones );B
  <A fan in |R^3>
  !gapprompt@gap>B !gapinput@T := ToricVariety( F );B
  <A toric variety of dimension 3>
  !gapprompt@gap>B !gapinput@[ IsSmooth( T ), IsComplete( T ), IsProjective( T ) ];B
  [ true, true, false ]
  !gapprompt@gap>B !gapinput@SRIdeal( T );B
  <A graded torsion-free (left) ideal given by 23 generators>
\end{Verbatim}
 

 }

 
\subsection{\textcolor{Chapter }{Convenient construction of toric varieties}}\label{Chapter_Toric_Varieties_Section_Toric_Varieties_Examples_Subsection_Convenient_construction_of_toric_varieties}
\logpage{[ 3, 1, 3 ]}
\hyperdef{L}{X83D1336D7AA5640E}{}
{
  
\begin{Verbatim}[commandchars=!@|,fontsize=\small,frame=single,label=Example]
  !gapprompt@gap>| !gapinput@rays := [ [1,0],[-1,0],[0,1],[0,-1] ];|
  [ [ 1, 0 ], [ -1, 0 ], [ 0, 1 ], [ 0, -1 ] ]
  !gapprompt@gap>| !gapinput@cones := [ [1,3],[1,4],[2,3],[2,4] ];|
  [ [1,3],[1,4],[2,3],[2,4] ]
  !gapprompt@gap>| !gapinput@weights := [ [1,0],[1,0],[0,1],[0,1] ];|
  [ [1,0],[1,0],[0,1],[0,1] ]
  !gapprompt@gap>| !gapinput@weights2 := [ [1,1],[1,1],[1,2],[1,2] ];|
  [ [1,1],[1,1],[1,2],[1,2] ]
  !gapprompt@gap>| !gapinput@tor1 := ToricVariety( rays, cones, weights, "x1,x2,y1,y2" );|
  <A toric variety of dimension 2>
  !gapprompt@gap>| !gapinput@CoxRing( tor1 );|
  Q[x2,y2,y1,x1]
  (weights: [ ( 1, 0 ), ( 0, 1 ), ( 0, 1 ), ( 1, 0 ) ])
  !gapprompt@gap>| !gapinput@tor2:= ToricVariety( rays, cones, weights, "q" );|
  <A toric variety of dimension 2>
  !gapprompt@gap>| !gapinput@CoxRing( tor2 );|
  Q[q_2,q_4,q_3,q_1]
  (weights: [ ( 1, 0 ), ( 0, 1 ), ( 0, 1 ), ( 1, 0 ) ])
  !gapprompt@gap>| !gapinput@tor3:= ToricVariety( rays, cones, weights );|
  <A toric variety of dimension 2>
  !gapprompt@gap>| !gapinput@CoxRing( tor3 );|
  Q[x_2,x_4,x_3,x_1]
  (weights: [ ( 1, 0 ), ( 0, 1 ), ( 0, 1 ), ( 1, 0 ) ])
  !gapprompt@gap>| !gapinput@tor4:= ToricVariety( rays, cones, weights2, "x1,x2,z1,z2" );|
  <A toric variety of dimension 2>
  !gapprompt@gap>| !gapinput@CoxRing( tor4 );|
  Q[x2,z2,z1,x1]
  (weights: [ ( 1, 1 ), ( 1, 2 ), ( 1, 2 ), ( 1, 1 ) ])
\end{Verbatim}
 

 }

 
\subsection{\textcolor{Chapter }{Toric varieties from gradings}}\label{Chapter_Toric_Varieties_Section_Toric_Varieties_Examples_Subsection_Toric_varieties_from_gradings}
\logpage{[ 3, 1, 4 ]}
\hyperdef{L}{X8130939E843A1494}{}
{
  The following example shows how to create the projective space $\mathbb{P}^2$ from the grading of its Cox ring. Note that this functionality requires the
package TopcomInterface. 
\begin{Verbatim}[commandchars=!@|,fontsize=\small,frame=single,label=Example]
  !gapprompt@gap>| !gapinput@g := [[1,1,1]];|
  [ [ 1,1,1 ] ]
  !gapprompt@gap>| !gapinput@v1 := ToricVarietyFromGrading( g );|
  <A toric variety of dimension 2>
  !gapprompt@gap>| !gapinput@CoxRing( v1 );|
  Q[x_1,x_2,x_3]
  (weights: [ 1, 1, 1 ])
\end{Verbatim}
 The following example shows how to create the resolved conifold(s) from the
grading of its Cox ring. 
\begin{Verbatim}[commandchars=!@|,fontsize=\small,frame=single,label=Example]
  !gapprompt@gap>| !gapinput@g2 := [[1,1,-1,-1]];|
  [ [ 1,1,-1,-1 ] ]
  !gapprompt@gap>| !gapinput@v2 := ToricVarietiesFromGrading( g2 );|
  [ <A toric variety of dimension 3>, <A toric variety of dimension 3> ]
  !gapprompt@gap>| !gapinput@CoxRing( v2[ 1 ] );|
  Q[x_1,x_2,x_3,x_4]
  (weights: [ 1, -1, -1, 1 ])
  !gapprompt@gap>| !gapinput@Display( SRIdeal( v2[ 1 ] ) );|
  x_2*x_3
  
  A (left) ideal generated by the entry of the above matrix
  
  (graded, degree of generator: -2)
  !gapprompt@gap>| !gapinput@Display( SRIdeal( v2[ 2 ] ) );|
  x_1*x_4
  
  A (left) ideal generated by the entry of the above matrix
  
  (graded, degree of generator: 2)
\end{Verbatim}
 

 }

 
\subsection{\textcolor{Chapter }{Blowups of toric varieties by star subdivisions of fans}}\label{Chapter_Toric_Varieties_Section_Toric_Varieties_Examples_Subsection_Blowups_of_toric_varieties_by_star_subdivisions_of_fans}
\logpage{[ 3, 1, 5 ]}
\hyperdef{L}{X815F92AA7A126BD5}{}
{
  The following code exemplifies blowups of the 3-dimensional affine space. 
\begin{Verbatim}[commandchars=!@D,fontsize=\small,frame=single,label=Example]
  !gapprompt@gap>D !gapinput@rays := [ [1,0,0], [0,1,0], [0,0,1] ];D
  [ [1,0,0], [0,1,0], [0,0,1] ]
  !gapprompt@gap>D !gapinput@max_cones := [ [1,2,3] ];D
  [ [1,2,3] ]
  !gapprompt@gap>D !gapinput@fan := Fan( rays, max_cones );D
  <A fan in |R^3>
  !gapprompt@gap>D !gapinput@C3 := ToricVariety( rays, max_cones, [[0],[0],[0]], "x1,x2,x3" );D
  <A toric variety of dimension 3>
  !gapprompt@gap>D !gapinput@B1C3 := BlowupOfToricVariety( C3, "x1,x2,x3", "u0" );D
  <A toric variety of dimension 3>
  !gapprompt@gap>D !gapinput@[ IsComplete( B1C3 ), IsOrbifold( B1C3 ), IsSmooth( B1C3 ) ];D
  [ false, true, true ]
  !gapprompt@gap>D !gapinput@B2C3 := BlowupOfToricVariety( B1C3, "x1,u0", "u1" );D
  <A toric variety of dimension 3>
  !gapprompt@gap>D !gapinput@Rank( ClassGroup( B2C3 ) );D
  3
  !gapprompt@gap>D !gapinput@B3C3 := BlowupOfToricVariety( B2C3, "x1,u1", "u2" );D
  <A toric variety of dimension 3>
  !gapprompt@gap>D !gapinput@CoxRing( B3C3 );D
  Q[x3,x2,x1,u0,u1,u2]
  (weights: [ ( 0, 1, 0, 0 ), ( 0, 1, 0, 0 ), ( 0, 1, 1, 1 ), 
  ( 0, -1, 1, 0 ), ( 0, 0, -1, 1 ), ( 0, 0, 0, -1 ) ])
\end{Verbatim}
 Likewise, we can also perform blowups of the 3-dimensional projective space. 
\begin{Verbatim}[commandchars=!@D,fontsize=\small,frame=single,label=Example]
  !gapprompt@gap>D !gapinput@rays := [ [1,0,0], [0,1,0], [0,0,1], [-1,-1,-1] ];D
  [ [1,0,0], [0,1,0], [0,0,1], [-1,-1,-1] ]
  !gapprompt@gap>D !gapinput@max_cones := [ [1,2,3], [1,2,4], [1,3,4], [2,3,4] ];D
  [ [1,2,3], [1,2,4], [1,3,4], [2,3,4] ]
  !gapprompt@gap>D !gapinput@fan := Fan( rays, max_cones );D
  <A fan in |R^3>
  !gapprompt@gap>D !gapinput@P3 := ToricVariety( rays, max_cones, [[1],[1],[1],[1]], "x1,x2,x3,x4" );D
  <A toric variety of dimension 3>
  !gapprompt@gap>D !gapinput@B1P3 := BlowupOfToricVariety( P3, "x1,x2,x3", "u0" );D
  <A toric variety of dimension 3>
  !gapprompt@gap>D !gapinput@[ IsComplete( B1P3 ), IsOrbifold( B1P3 ), IsSmooth( B1P3 ) ];D
  [ true, true, true ]
  !gapprompt@gap>D !gapinput@B2P3 := BlowupOfToricVariety( B1P3, "x1,u0", "u1" );D
  <A toric variety of dimension 3>
  !gapprompt@gap>D !gapinput@Rank( ClassGroup( B2C3 ) );D
  3
  !gapprompt@gap>D !gapinput@B3P3 := BlowupOfToricVariety( B2P3, "x1,u1", "u2" );D
  <A toric variety of dimension 3>
  !gapprompt@gap>D !gapinput@CoxRing( B3P3 );D
  Q[x4,x3,x2,x1,u0,u1,u2]
  (weights: [ ( 1, 0, 0, 0 ), ( 1, 1, 0, 0 ), ( 1, 1, 0, 0 ), 
  ( 1, 1, 1, 1 ), ( 0, -1, 1, 0 ), ( 0, 0, -1, 1 ), ( 0, 0, 0, -1 ) ])
\end{Verbatim}
 Also, we can perform blowups of a generalized Hirzebruch 3-fold. 
\begin{Verbatim}[commandchars=!@|,fontsize=\small,frame=single,label=Example]
  !gapprompt@gap>| !gapinput@vars := "u,s,v,t,r";|
  "u,s,v,t,r"
  !gapprompt@gap>| !gapinput@rays := [ [0,0,-1],[1,0,0],[0,1,0],[-1,-1,-17],[0,0,1] ];|
  [ [0,0,-1],[1,0,0],[0,1,0],[-1,-1,-17],[0,0,1] ]
  !gapprompt@gap>| !gapinput@cones := [ [1,2,3], [1,2,4], [1,3,4], [2,3,5], [2,4,5], [3,4,5] ];|
  [ [1,2,3], [1,2,4], [1,3,4], [2,3,5], [2,4,5], [3,4,5] ]
  !gapprompt@gap>| !gapinput@weights := [ [1,-17], [0,1], [0,1], [0,1], [1,0] ];|
  [ [1,-17], [0,1], [0,1], [0,1], [1,0] ]
  !gapprompt@gap>| !gapinput@H3fold := ToricVariety( rays, cones, weights, vars );|
  <A toric variety of dimension 3>
  !gapprompt@gap>| !gapinput@B1H3fold := BlowupOfToricVariety( H3fold, "u,s", "u1" );|
  <A toric variety of dimension 3>
  !gapprompt@gap>| !gapinput@CoxRing( B1H3fold );|
  Q[t,u,r,v,u1,s]
  (weights: [ ( 0, 1, 0 ), ( 1, -17, 1 ), ( 1, 0, 0 ), 
  ( 0, 1, 0 ), ( 0, 0, -1 ), ( 0, 1, 1 ) ])
\end{Verbatim}
 This example easily extends to an entire sequence of blowups. 
\begin{Verbatim}[commandchars=!@|,fontsize=\small,frame=single,label=Example]
  !gapprompt@gap>| !gapinput@vars := "u,s,v,t,r,x,y,w";|
  "u,s,v,t,r,x,y,w"
  !gapprompt@gap>| !gapinput@rays := [ [0,0,-1,-2,-3], [1,0,0,-2,-3], [0,1,0,-2,-3], [-1,-1,-17,-2,-3], |
  !gapprompt@>| !gapinput@          [0,0,1,-2,-3], [0, 0, 0, 1, 0], |
  !gapprompt@>| !gapinput@[0, 0, 0, 0, 1], [0, 0, 0, -2, -3] ];|
  [ [0,0,-1,-2,-3], [1,0,0,-2,-3], [0,1,0,-2,-3], [-1,-1,-17,-2,-3], 
  [0,0,1,-2,-3], [0, 0, 0, 1, 0], [0, 0, 0, 0, 1], [0, 0, 0, -2, -3] ]
  !gapprompt@gap>| !gapinput@cones := [ [1,2,3,6,7], [1,2,3,6,8], [1,2,3,7,8], [1,2,4,6,7], [1,2,4,6,8],|
  !gapprompt@>| !gapinput@           [1,2,4,7,8], [1,3,4,6,7], [1,3,4,6,8], [1,3,4,7,8], [2,3,5,6,7],|
  !gapprompt@>| !gapinput@           [2,3,5,6,8], [2,3,5,7,8], [2,4,5,6,7], [2,4,5,6,8], [2,4,5,7,8],|
  !gapprompt@>| !gapinput@           [3,4,5,6,7], [3,4,5,6,8], [3,4,5,7,8] ];|
  [ [ 1, 2, 3, 6, 7 ], [ 1, 2, 3, 6, 8 ], [ 1, 2, 3, 7, 8 ],
    [ 1, 2, 4, 6, 7 ], [ 1, 2, 4, 6, 8 ], [ 1, 2, 4, 7, 8 ],
    [ 1, 3, 4, 6, 7 ], [ 1, 3, 4, 6, 8 ], [ 1, 3, 4, 7, 8 ],
    [ 2, 3, 5, 6, 7 ], [ 2, 3, 5, 6, 8 ], [ 2, 3, 5, 7, 8 ],
    [ 2, 4, 5, 6, 7 ], [ 2, 4, 5, 6, 8 ], [ 2, 4, 5, 7, 8 ],
    [ 3, 4, 5, 6, 7 ], [ 3, 4, 5, 6, 8 ], [ 3, 4, 5, 7, 8 ] ]
  !gapprompt@gap>| !gapinput@w := [ [1,-17,0], [0,1,0], [0,1,0], [0,1,0], [1,0,0], [0,0,2], [0,0,3], |
  !gapprompt@>| !gapinput@       [-2,14,1] ];|
  [ [1,-17,0], [0,1,0], [0,1,0], [0,1,0], [1,0,0], [0,0,2], [0,0,3], [-2,14,1] ]
  !gapprompt@gap>| !gapinput@base := ToricVariety( rays, cones, w, vars );|
  <A toric variety of dimension 5>
  !gapprompt@gap>| !gapinput@b1 := BlowupOfToricVariety( base, "x,y,u", "u1" );|
  <A toric variety of dimension 5>
  !gapprompt@gap>| !gapinput@b2 := BlowupOfToricVariety( b1, "x,y,u1", "u2" );|
  <A toric variety of dimension 5>
  !gapprompt@gap>| !gapinput@b3 := BlowupOfToricVariety( b2, "y,u1", "u3" );|
  <A toric variety of dimension 5>
  !gapprompt@gap>| !gapinput@b4 := BlowupOfToricVariety( b3, "y,u2", "u4" );|
  <A toric variety of dimension 5>
  !gapprompt@gap>| !gapinput@b5 := BlowupOfToricVariety( b4, "u2,u3", "u5" );|
  <A toric variety of dimension 5>
  !gapprompt@gap>| !gapinput@b6 := BlowupOfToricVariety( b5, "u1,u3", "u6" );|
  <A toric variety of dimension 5>
  !gapprompt@gap>| !gapinput@b7 := BlowupOfToricVariety( b6, "u2,u4", "u7" );|
  <A toric variety of dimension 5>
  !gapprompt@gap>| !gapinput@b8 := BlowupOfToricVariety( b7, "u3,u4", "u8" );|
  <A toric variety of dimension 5>
  !gapprompt@gap>| !gapinput@b9 := BlowupOfToricVariety( b8, "u4,u5", "u9" );|
  <A toric variety of dimension 5>
  !gapprompt@gap>| !gapinput@b10 := BlowupOfToricVariety( b9, "u5,u8", "u10" );|
  <A toric variety of dimension 5>
  !gapprompt@gap>| !gapinput@b11 := BlowupOfToricVariety( b10, "u4,u8", "u11" );|
  <A toric variety of dimension 5>
  !gapprompt@gap>| !gapinput@b12 := BlowupOfToricVariety( b11, "u4,u9", "u12" );|
  <A toric variety of dimension 5>
  !gapprompt@gap>| !gapinput@b13 := BlowupOfToricVariety( b12, "u8,u9", "u13" );|
  <A toric variety of dimension 5>
  !gapprompt@gap>| !gapinput@b14 := BlowupOfToricVariety( b13, "u9,u11", "u14" );|
  <A toric variety of dimension 5>
  !gapprompt@gap>| !gapinput@b15 := BlowupOfToricVariety( b14, "u4,v", "d" );|
  <A toric variety of dimension 5>
  !gapprompt@gap>| !gapinput@final_space := BlowupOfToricVariety( b15, "u3,u5", "u15" );|
  <A toric variety of dimension 5>
\end{Verbatim}
 This sequence of blowups can also be performed with a single command. 
\begin{Verbatim}[commandchars=!@|,fontsize=\small,frame=single,label=Example]
  !gapprompt@gap>| !gapinput@final_space2 := SequenceOfBlowupsOfToricVariety( base, |
  !gapprompt@>| !gapinput@                    [ ["x,y,u","u1"], |
  !gapprompt@>| !gapinput@                    ["x,y,u1","u2"],|
  !gapprompt@>| !gapinput@                    ["y,u1","u3"],|
  !gapprompt@>| !gapinput@                    ["y,u2","u4"],|
  !gapprompt@>| !gapinput@                    ["u2,u3","u5"],|
  !gapprompt@>| !gapinput@                    ["u1,u3","u6"],|
  !gapprompt@>| !gapinput@                    ["u2,u4","u7"],|
  !gapprompt@>| !gapinput@                    ["u3,u4","u8"],|
  !gapprompt@>| !gapinput@                    ["u4,u5","u9"],|
  !gapprompt@>| !gapinput@                    ["u5,u8","u10"],|
  !gapprompt@>| !gapinput@                    ["u4,u8","u11"],|
  !gapprompt@>| !gapinput@                    ["u4,u9","u12"],|
  !gapprompt@>| !gapinput@                    ["u8,u9","u13"],|
  !gapprompt@>| !gapinput@                    ["u9,u11","u14"],|
  !gapprompt@>| !gapinput@                    ["u4,v","d"],|
  !gapprompt@>| !gapinput@                    ["u3,u5","u15"] ] );|
  <A toric variety of dimension 5>
  !gapprompt@gap>| !gapinput@[ IsComplete( final_space2 ), IsOrbifold( final_space2 ), |
  !gapprompt@>| !gapinput@  IsSmooth( final_space2 ) ];|
  [ true, true, false ]
\end{Verbatim}
 

 }

 }

 
\section{\textcolor{Chapter }{Toric variety: Category and Representations}}\label{Chapter_Toric_Varieties_Section_Toric_variety_Category_and_Representations}
\logpage{[ 3, 2, 0 ]}
\hyperdef{L}{X8108B9978021989B}{}
{
  

\subsection{\textcolor{Chapter }{IsToricVariety (for IsObject)}}
\logpage{[ 3, 2, 1 ]}\nobreak
\hyperdef{L}{X825DA516855326E1}{}
{\noindent\textcolor{FuncColor}{$\triangleright$\enspace\texttt{IsToricVariety({\mdseries\slshape M})\index{IsToricVariety@\texttt{IsToricVariety}!for IsObject}
\label{IsToricVariety:for IsObject}
}\hfill{\scriptsize (filter)}}\\
\textbf{\indent Returns:\ }
true or false 



 Checks if an object is a toric variety. }

 

\subsection{\textcolor{Chapter }{IsCategoryOfToricVarieties (for IsHomalgCategory)}}
\logpage{[ 3, 2, 2 ]}\nobreak
\hyperdef{L}{X87C8FCC6816CD49C}{}
{\noindent\textcolor{FuncColor}{$\triangleright$\enspace\texttt{IsCategoryOfToricVarieties({\mdseries\slshape object})\index{IsCategoryOfToricVarieties@\texttt{IsCategoryOfToricVarieties}!for IsHomalgCategory}
\label{IsCategoryOfToricVarieties:for IsHomalgCategory}
}\hfill{\scriptsize (filter)}}\\
\textbf{\indent Returns:\ }
\texttt{true} or \texttt{false} 



 The \mbox{\texttt{\mdseries\slshape GAP}} category of toric varieties. }

 

\subsection{\textcolor{Chapter }{twitter (for IsToricVariety)}}
\logpage{[ 3, 2, 3 ]}\nobreak
\hyperdef{L}{X82DC51717AD9628C}{}
{\noindent\textcolor{FuncColor}{$\triangleright$\enspace\texttt{twitter({\mdseries\slshape vari})\index{twitter@\texttt{twitter}!for IsToricVariety}
\label{twitter:for IsToricVariety}
}\hfill{\scriptsize (attribute)}}\\
\textbf{\indent Returns:\ }
a ring 



 This is a dummy to get immediate methods triggered at some times. It never has
a value. }

 }

 
\section{\textcolor{Chapter }{Properties}}\label{Chapter_Toric_Varieties_Section_Properties}
\logpage{[ 3, 3, 0 ]}
\hyperdef{L}{X871597447BB998A1}{}
{
  

\subsection{\textcolor{Chapter }{IsNormalVariety (for IsToricVariety)}}
\logpage{[ 3, 3, 1 ]}\nobreak
\hyperdef{L}{X7BB01CF579A454D7}{}
{\noindent\textcolor{FuncColor}{$\triangleright$\enspace\texttt{IsNormalVariety({\mdseries\slshape vari})\index{IsNormalVariety@\texttt{IsNormalVariety}!for IsToricVariety}
\label{IsNormalVariety:for IsToricVariety}
}\hfill{\scriptsize (property)}}\\
\textbf{\indent Returns:\ }
true or false 



 Checks if the toric variety \mbox{\texttt{\mdseries\slshape vari}} is a normal variety. }

 

\subsection{\textcolor{Chapter }{IsAffine (for IsToricVariety)}}
\logpage{[ 3, 3, 2 ]}\nobreak
\hyperdef{L}{X821D89E687652A94}{}
{\noindent\textcolor{FuncColor}{$\triangleright$\enspace\texttt{IsAffine({\mdseries\slshape vari})\index{IsAffine@\texttt{IsAffine}!for IsToricVariety}
\label{IsAffine:for IsToricVariety}
}\hfill{\scriptsize (property)}}\\
\textbf{\indent Returns:\ }
true or false 



 Checks if the toric variety \mbox{\texttt{\mdseries\slshape vari}} is an affine variety. }

 

\subsection{\textcolor{Chapter }{IsProjective (for IsToricVariety)}}
\logpage{[ 3, 3, 3 ]}\nobreak
\hyperdef{L}{X7C408D7B7E064821}{}
{\noindent\textcolor{FuncColor}{$\triangleright$\enspace\texttt{IsProjective({\mdseries\slshape vari})\index{IsProjective@\texttt{IsProjective}!for IsToricVariety}
\label{IsProjective:for IsToricVariety}
}\hfill{\scriptsize (property)}}\\
\textbf{\indent Returns:\ }
true or false 



 Checks if the toric variety \mbox{\texttt{\mdseries\slshape vari}} is a projective variety. }

 

\subsection{\textcolor{Chapter }{IsSmooth (for IsToricVariety)}}
\logpage{[ 3, 3, 4 ]}\nobreak
\hyperdef{L}{X7B602E5F83D130E6}{}
{\noindent\textcolor{FuncColor}{$\triangleright$\enspace\texttt{IsSmooth({\mdseries\slshape vari})\index{IsSmooth@\texttt{IsSmooth}!for IsToricVariety}
\label{IsSmooth:for IsToricVariety}
}\hfill{\scriptsize (property)}}\\
\textbf{\indent Returns:\ }
true or false 



 Checks if the toric variety \mbox{\texttt{\mdseries\slshape vari}} is smooth. }

 

\subsection{\textcolor{Chapter }{IsComplete (for IsToricVariety)}}
\logpage{[ 3, 3, 5 ]}\nobreak
\hyperdef{L}{X7FFBADCC7BB02030}{}
{\noindent\textcolor{FuncColor}{$\triangleright$\enspace\texttt{IsComplete({\mdseries\slshape vari})\index{IsComplete@\texttt{IsComplete}!for IsToricVariety}
\label{IsComplete:for IsToricVariety}
}\hfill{\scriptsize (property)}}\\
\textbf{\indent Returns:\ }
true or false 



 Checks if the toric variety \mbox{\texttt{\mdseries\slshape vari}} is complete. }

 

\subsection{\textcolor{Chapter }{HasTorusfactor (for IsToricVariety)}}
\logpage{[ 3, 3, 6 ]}\nobreak
\hyperdef{L}{X7B50D0928436C668}{}
{\noindent\textcolor{FuncColor}{$\triangleright$\enspace\texttt{HasTorusfactor({\mdseries\slshape vari})\index{HasTorusfactor@\texttt{HasTorusfactor}!for IsToricVariety}
\label{HasTorusfactor:for IsToricVariety}
}\hfill{\scriptsize (property)}}\\
\textbf{\indent Returns:\ }
true or false 



 Checks if the toric variety \mbox{\texttt{\mdseries\slshape vari}} has a torus factor. }

 

\subsection{\textcolor{Chapter }{HasNoTorusfactor (for IsToricVariety)}}
\logpage{[ 3, 3, 7 ]}\nobreak
\hyperdef{L}{X7C48A0497D35EA70}{}
{\noindent\textcolor{FuncColor}{$\triangleright$\enspace\texttt{HasNoTorusfactor({\mdseries\slshape vari})\index{HasNoTorusfactor@\texttt{HasNoTorusfactor}!for IsToricVariety}
\label{HasNoTorusfactor:for IsToricVariety}
}\hfill{\scriptsize (property)}}\\
\textbf{\indent Returns:\ }
true or false 



 Checks if the toric variety \mbox{\texttt{\mdseries\slshape vari}} has no torus factor. }

 

\subsection{\textcolor{Chapter }{IsOrbifold (for IsToricVariety)}}
\logpage{[ 3, 3, 8 ]}\nobreak
\hyperdef{L}{X7D1DE31A87B88797}{}
{\noindent\textcolor{FuncColor}{$\triangleright$\enspace\texttt{IsOrbifold({\mdseries\slshape vari})\index{IsOrbifold@\texttt{IsOrbifold}!for IsToricVariety}
\label{IsOrbifold:for IsToricVariety}
}\hfill{\scriptsize (property)}}\\
\textbf{\indent Returns:\ }
true or false 



 Checks if the toric variety \mbox{\texttt{\mdseries\slshape vari}} has an orbifold, which is, in the toric case, equivalent to the simpliciality
of the fan. }

 

\subsection{\textcolor{Chapter }{IsSimplicial (for IsToricVariety)}}
\logpage{[ 3, 3, 9 ]}\nobreak
\hyperdef{L}{X86DC51D486B2A033}{}
{\noindent\textcolor{FuncColor}{$\triangleright$\enspace\texttt{IsSimplicial({\mdseries\slshape vari})\index{IsSimplicial@\texttt{IsSimplicial}!for IsToricVariety}
\label{IsSimplicial:for IsToricVariety}
}\hfill{\scriptsize (property)}}\\
\textbf{\indent Returns:\ }
true or false 



 Checks if the toric variety \mbox{\texttt{\mdseries\slshape vari}} is simplicial. This is a convenience method equivalent to IsOrbifold. }

 }

 
\section{\textcolor{Chapter }{Attributes}}\label{Chapter_Toric_Varieties_Section_Attributes}
\logpage{[ 3, 4, 0 ]}
\hyperdef{L}{X7C701DBF7BAE649A}{}
{
  

\subsection{\textcolor{Chapter }{AffineOpenCovering (for IsToricVariety)}}
\logpage{[ 3, 4, 1 ]}\nobreak
\hyperdef{L}{X87E5C0868418F3D0}{}
{\noindent\textcolor{FuncColor}{$\triangleright$\enspace\texttt{AffineOpenCovering({\mdseries\slshape vari})\index{AffineOpenCovering@\texttt{AffineOpenCovering}!for IsToricVariety}
\label{AffineOpenCovering:for IsToricVariety}
}\hfill{\scriptsize (attribute)}}\\
\textbf{\indent Returns:\ }
a list 



 Returns a torus invariant affine open covering of the variety \mbox{\texttt{\mdseries\slshape vari}}. The affine open cover is computed out of the cones of the fan. }

 

\subsection{\textcolor{Chapter }{CoxRing (for IsToricVariety)}}
\logpage{[ 3, 4, 2 ]}\nobreak
\hyperdef{L}{X7F65180D85380B35}{}
{\noindent\textcolor{FuncColor}{$\triangleright$\enspace\texttt{CoxRing({\mdseries\slshape vari})\index{CoxRing@\texttt{CoxRing}!for IsToricVariety}
\label{CoxRing:for IsToricVariety}
}\hfill{\scriptsize (attribute)}}\\
\textbf{\indent Returns:\ }
a ring 



 Returns the Cox ring of the variety \mbox{\texttt{\mdseries\slshape vari}}. The actual method requires a string with a name for the variables. A method
for computing the Cox ring without a variable given is not implemented. You
will get an error. }

 

\subsection{\textcolor{Chapter }{ListOfVariablesOfCoxRing (for IsToricVariety)}}
\logpage{[ 3, 4, 3 ]}\nobreak
\hyperdef{L}{X868FA9D6867D8A3B}{}
{\noindent\textcolor{FuncColor}{$\triangleright$\enspace\texttt{ListOfVariablesOfCoxRing({\mdseries\slshape vari})\index{ListOfVariablesOfCoxRing@\texttt{ListOfVariablesOfCoxRing}!for IsToricVariety}
\label{ListOfVariablesOfCoxRing:for IsToricVariety}
}\hfill{\scriptsize (attribute)}}\\
\textbf{\indent Returns:\ }
a list 



 Returns a list of the variables of the cox ring of the variety \mbox{\texttt{\mdseries\slshape vari}}. }

 

\subsection{\textcolor{Chapter }{ClassGroup (for IsToricVariety)}}
\logpage{[ 3, 4, 4 ]}\nobreak
\hyperdef{L}{X7F35787D83E1866B}{}
{\noindent\textcolor{FuncColor}{$\triangleright$\enspace\texttt{ClassGroup({\mdseries\slshape vari})\index{ClassGroup@\texttt{ClassGroup}!for IsToricVariety}
\label{ClassGroup:for IsToricVariety}
}\hfill{\scriptsize (attribute)}}\\
\textbf{\indent Returns:\ }
a module 



 Returns the class group of the variety \mbox{\texttt{\mdseries\slshape vari}} as factor of a free module. }

 

\subsection{\textcolor{Chapter }{TorusInvariantDivisorGroup (for IsToricVariety)}}
\logpage{[ 3, 4, 5 ]}\nobreak
\hyperdef{L}{X86353AF986EDFEB9}{}
{\noindent\textcolor{FuncColor}{$\triangleright$\enspace\texttt{TorusInvariantDivisorGroup({\mdseries\slshape vari})\index{TorusInvariantDivisorGroup@\texttt{TorusInvariantDivisorGroup}!for IsToricVariety}
\label{TorusInvariantDivisorGroup:for IsToricVariety}
}\hfill{\scriptsize (attribute)}}\\
\textbf{\indent Returns:\ }
a module 



 Returns the subgroup of the Weil divisor group of the variety \mbox{\texttt{\mdseries\slshape vari}} generated by the torus invariant prime divisors. This is always a finitely
generated free module over the integers. }

 

\subsection{\textcolor{Chapter }{MapFromCharacterToPrincipalDivisor (for IsToricVariety)}}
\logpage{[ 3, 4, 6 ]}\nobreak
\hyperdef{L}{X7F2F17E382730BCF}{}
{\noindent\textcolor{FuncColor}{$\triangleright$\enspace\texttt{MapFromCharacterToPrincipalDivisor({\mdseries\slshape vari})\index{MapFromCharacterToPrincipalDivisor@\texttt{MapFromCharacterToPrincipalDivisor}!for IsToricVariety}
\label{MapFromCharacterToPrincipalDivisor:for IsToricVariety}
}\hfill{\scriptsize (attribute)}}\\
\textbf{\indent Returns:\ }
a morphism 



 Returns a map which maps an element of the character group into the torus
invariant Weil group of the variety \mbox{\texttt{\mdseries\slshape vari}}. This has to be viewed as a help method to compute divisor classes. }

 

\subsection{\textcolor{Chapter }{MapFromWeilDivisorsToClassGroup (for IsToricVariety)}}
\logpage{[ 3, 4, 7 ]}\nobreak
\hyperdef{L}{X7FF75DB77EF5600A}{}
{\noindent\textcolor{FuncColor}{$\triangleright$\enspace\texttt{MapFromWeilDivisorsToClassGroup({\mdseries\slshape vari})\index{MapFromWeilDivisorsToClassGroup@\texttt{MapFromWeilDivisorsToClassGroup}!for IsToricVariety}
\label{MapFromWeilDivisorsToClassGroup:for IsToricVariety}
}\hfill{\scriptsize (attribute)}}\\
\textbf{\indent Returns:\ }
a morphism 



 Returns a map which maps a Weil divisor into the class group. }

 

\subsection{\textcolor{Chapter }{Dimension (for IsToricVariety)}}
\logpage{[ 3, 4, 8 ]}\nobreak
\hyperdef{L}{X80ECD8BA7FF3CD96}{}
{\noindent\textcolor{FuncColor}{$\triangleright$\enspace\texttt{Dimension({\mdseries\slshape vari})\index{Dimension@\texttt{Dimension}!for IsToricVariety}
\label{Dimension:for IsToricVariety}
}\hfill{\scriptsize (attribute)}}\\
\textbf{\indent Returns:\ }
an integer 



 Returns the dimension of the variety \mbox{\texttt{\mdseries\slshape vari}}. }

 

\subsection{\textcolor{Chapter }{DimensionOfTorusfactor (for IsToricVariety)}}
\logpage{[ 3, 4, 9 ]}\nobreak
\hyperdef{L}{X85336FDD8728605B}{}
{\noindent\textcolor{FuncColor}{$\triangleright$\enspace\texttt{DimensionOfTorusfactor({\mdseries\slshape vari})\index{DimensionOfTorusfactor@\texttt{DimensionOfTorusfactor}!for IsToricVariety}
\label{DimensionOfTorusfactor:for IsToricVariety}
}\hfill{\scriptsize (attribute)}}\\
\textbf{\indent Returns:\ }
an integer 



 Returns the dimension of the torus factor of the variety \mbox{\texttt{\mdseries\slshape vari}}. }

 

\subsection{\textcolor{Chapter }{CoordinateRingOfTorus (for IsToricVariety)}}
\logpage{[ 3, 4, 10 ]}\nobreak
\hyperdef{L}{X84874BCA7C941D76}{}
{\noindent\textcolor{FuncColor}{$\triangleright$\enspace\texttt{CoordinateRingOfTorus({\mdseries\slshape vari})\index{CoordinateRingOfTorus@\texttt{CoordinateRingOfTorus}!for IsToricVariety}
\label{CoordinateRingOfTorus:for IsToricVariety}
}\hfill{\scriptsize (attribute)}}\\
\textbf{\indent Returns:\ }
a ring 



 Returns the coordinate ring of the torus of the variety \mbox{\texttt{\mdseries\slshape vari}}. This is by default done with the variables \mbox{\texttt{\mdseries\slshape x1}} to \mbox{\texttt{\mdseries\slshape xn}} where \mbox{\texttt{\mdseries\slshape n}} is the dimension of the variety. To use a different set of variables, a
convenience method is provided and described in the \mbox{\texttt{\mdseries\slshape methods}} section. }

 

\subsection{\textcolor{Chapter }{ListOfVariablesOfCoordinateRingOfTorus (for IsToricVariety)}}
\logpage{[ 3, 4, 11 ]}\nobreak
\hyperdef{L}{X7DA644E9799B9EC4}{}
{\noindent\textcolor{FuncColor}{$\triangleright$\enspace\texttt{ListOfVariablesOfCoordinateRingOfTorus({\mdseries\slshape vari})\index{ListOfVariablesOfCoordinateRingOfTorus@\texttt{List}\-\texttt{Of}\-\texttt{Variables}\-\texttt{Of}\-\texttt{Coordinate}\-\texttt{Ring}\-\texttt{Of}\-\texttt{Torus}!for IsToricVariety}
\label{ListOfVariablesOfCoordinateRingOfTorus:for IsToricVariety}
}\hfill{\scriptsize (attribute)}}\\
\textbf{\indent Returns:\ }
a list 



 Returns the list of variables in the coordinate ring of the torus of the
variety \mbox{\texttt{\mdseries\slshape vari}}. }

 

\subsection{\textcolor{Chapter }{IsProductOf (for IsToricVariety)}}
\logpage{[ 3, 4, 12 ]}\nobreak
\hyperdef{L}{X872889BB7A8C950E}{}
{\noindent\textcolor{FuncColor}{$\triangleright$\enspace\texttt{IsProductOf({\mdseries\slshape vari})\index{IsProductOf@\texttt{IsProductOf}!for IsToricVariety}
\label{IsProductOf:for IsToricVariety}
}\hfill{\scriptsize (attribute)}}\\
\textbf{\indent Returns:\ }
a list 



 If the variety \mbox{\texttt{\mdseries\slshape vari}} is a product of 2 or more varieties, the list contains those varieties. If it
is not a product or at least not generated as a product, the list only
contains the variety itself. }

 

\subsection{\textcolor{Chapter }{CharacterLattice (for IsToricVariety)}}
\logpage{[ 3, 4, 13 ]}\nobreak
\hyperdef{L}{X7D5F531086B349DC}{}
{\noindent\textcolor{FuncColor}{$\triangleright$\enspace\texttt{CharacterLattice({\mdseries\slshape vari})\index{CharacterLattice@\texttt{CharacterLattice}!for IsToricVariety}
\label{CharacterLattice:for IsToricVariety}
}\hfill{\scriptsize (attribute)}}\\
\textbf{\indent Returns:\ }
a module 



 The method returns the character lattice of the variety \mbox{\texttt{\mdseries\slshape vari}}, computed as the containing grid of the underlying convex object, if it
exists. }

 

\subsection{\textcolor{Chapter }{TorusInvariantPrimeDivisors (for IsToricVariety)}}
\logpage{[ 3, 4, 14 ]}\nobreak
\hyperdef{L}{X7F3E55DB87BF1B9D}{}
{\noindent\textcolor{FuncColor}{$\triangleright$\enspace\texttt{TorusInvariantPrimeDivisors({\mdseries\slshape vari})\index{TorusInvariantPrimeDivisors@\texttt{TorusInvariantPrimeDivisors}!for IsToricVariety}
\label{TorusInvariantPrimeDivisors:for IsToricVariety}
}\hfill{\scriptsize (attribute)}}\\
\textbf{\indent Returns:\ }
a list 



 The method returns a list of the torus invariant prime divisors of the variety \mbox{\texttt{\mdseries\slshape vari}}. }

 

\subsection{\textcolor{Chapter }{IrrelevantIdeal (for IsToricVariety)}}
\logpage{[ 3, 4, 15 ]}\nobreak
\hyperdef{L}{X84FE626E7A333413}{}
{\noindent\textcolor{FuncColor}{$\triangleright$\enspace\texttt{IrrelevantIdeal({\mdseries\slshape vari})\index{IrrelevantIdeal@\texttt{IrrelevantIdeal}!for IsToricVariety}
\label{IrrelevantIdeal:for IsToricVariety}
}\hfill{\scriptsize (attribute)}}\\
\textbf{\indent Returns:\ }
an ideal 



 Returns the irrelevant ideal of the Cox ring of the variety \mbox{\texttt{\mdseries\slshape vari}}. }

 

\subsection{\textcolor{Chapter }{SRIdeal (for IsToricVariety)}}
\logpage{[ 3, 4, 16 ]}\nobreak
\hyperdef{L}{X7FF5AFAF7EF6C050}{}
{\noindent\textcolor{FuncColor}{$\triangleright$\enspace\texttt{SRIdeal({\mdseries\slshape vari})\index{SRIdeal@\texttt{SRIdeal}!for IsToricVariety}
\label{SRIdeal:for IsToricVariety}
}\hfill{\scriptsize (attribute)}}\\
\textbf{\indent Returns:\ }
an ideal 



 Returns the Stanley-Rei{\ss}ner ideal of the Cox ring of the variety \mbox{\texttt{\mdseries\slshape vari}}. }

 

\subsection{\textcolor{Chapter }{MorphismFromCoxVariety (for IsToricVariety)}}
\logpage{[ 3, 4, 17 ]}\nobreak
\hyperdef{L}{X78EDFFFA846DA36B}{}
{\noindent\textcolor{FuncColor}{$\triangleright$\enspace\texttt{MorphismFromCoxVariety({\mdseries\slshape vari})\index{MorphismFromCoxVariety@\texttt{MorphismFromCoxVariety}!for IsToricVariety}
\label{MorphismFromCoxVariety:for IsToricVariety}
}\hfill{\scriptsize (attribute)}}\\
\textbf{\indent Returns:\ }
a morphism 



 The method returns the quotient morphism from the variety of the Cox ring to
the variety \mbox{\texttt{\mdseries\slshape vari}}. }

 

\subsection{\textcolor{Chapter }{CoxVariety (for IsToricVariety)}}
\logpage{[ 3, 4, 18 ]}\nobreak
\hyperdef{L}{X816182EE86750E10}{}
{\noindent\textcolor{FuncColor}{$\triangleright$\enspace\texttt{CoxVariety({\mdseries\slshape vari})\index{CoxVariety@\texttt{CoxVariety}!for IsToricVariety}
\label{CoxVariety:for IsToricVariety}
}\hfill{\scriptsize (attribute)}}\\
\textbf{\indent Returns:\ }
a variety 



 The method returns the Cox variety of the variety \mbox{\texttt{\mdseries\slshape vari}}. }

 

\subsection{\textcolor{Chapter }{FanOfVariety (for IsToricVariety)}}
\logpage{[ 3, 4, 19 ]}\nobreak
\hyperdef{L}{X7EF6A88D79773074}{}
{\noindent\textcolor{FuncColor}{$\triangleright$\enspace\texttt{FanOfVariety({\mdseries\slshape vari})\index{FanOfVariety@\texttt{FanOfVariety}!for IsToricVariety}
\label{FanOfVariety:for IsToricVariety}
}\hfill{\scriptsize (attribute)}}\\
\textbf{\indent Returns:\ }
a fan 



 Returns the fan of the variety \mbox{\texttt{\mdseries\slshape vari}}. This is set by default. }

 

\subsection{\textcolor{Chapter }{CartierTorusInvariantDivisorGroup (for IsToricVariety)}}
\logpage{[ 3, 4, 20 ]}\nobreak
\hyperdef{L}{X7FA1D58881BC17AF}{}
{\noindent\textcolor{FuncColor}{$\triangleright$\enspace\texttt{CartierTorusInvariantDivisorGroup({\mdseries\slshape vari})\index{CartierTorusInvariantDivisorGroup@\texttt{CartierTorusInvariantDivisorGroup}!for IsToricVariety}
\label{CartierTorusInvariantDivisorGroup:for IsToricVariety}
}\hfill{\scriptsize (attribute)}}\\
\textbf{\indent Returns:\ }
a module 



 Returns the the group of Cartier divisors of the variety \mbox{\texttt{\mdseries\slshape vari}} as a subgroup of the divisor group. }

 

\subsection{\textcolor{Chapter }{PicardGroup (for IsToricVariety)}}
\logpage{[ 3, 4, 21 ]}\nobreak
\hyperdef{L}{X7D5508407A039CBC}{}
{\noindent\textcolor{FuncColor}{$\triangleright$\enspace\texttt{PicardGroup({\mdseries\slshape vari})\index{PicardGroup@\texttt{PicardGroup}!for IsToricVariety}
\label{PicardGroup:for IsToricVariety}
}\hfill{\scriptsize (attribute)}}\\
\textbf{\indent Returns:\ }
a module 



 Returns the Picard group of the variety \mbox{\texttt{\mdseries\slshape vari}} as factor of a free module. }

 

\subsection{\textcolor{Chapter }{NameOfVariety (for IsToricVariety)}}
\logpage{[ 3, 4, 22 ]}\nobreak
\hyperdef{L}{X7D964D0B7A315F74}{}
{\noindent\textcolor{FuncColor}{$\triangleright$\enspace\texttt{NameOfVariety({\mdseries\slshape vari})\index{NameOfVariety@\texttt{NameOfVariety}!for IsToricVariety}
\label{NameOfVariety:for IsToricVariety}
}\hfill{\scriptsize (attribute)}}\\
\textbf{\indent Returns:\ }
a string 



 Returns the name of the variety \mbox{\texttt{\mdseries\slshape vari}} if it has one and it is known or can be computed. }

 

\subsection{\textcolor{Chapter }{ZariskiCotangentSheaf (for IsToricVariety)}}
\logpage{[ 3, 4, 23 ]}\nobreak
\hyperdef{L}{X7A1AEE8A8531B014}{}
{\noindent\textcolor{FuncColor}{$\triangleright$\enspace\texttt{ZariskiCotangentSheaf({\mdseries\slshape vari})\index{ZariskiCotangentSheaf@\texttt{ZariskiCotangentSheaf}!for IsToricVariety}
\label{ZariskiCotangentSheaf:for IsToricVariety}
}\hfill{\scriptsize (attribute)}}\\
\textbf{\indent Returns:\ }
a f.p. graded \mbox{\texttt{\mdseries\slshape S}}-module 



 This method returns a f. p. graded \mbox{\texttt{\mdseries\slshape S}}-module (\mbox{\texttt{\mdseries\slshape S}} being the Cox ring of the variety), such that the sheafification of this
module is the Zariski cotangent sheaf of \mbox{\texttt{\mdseries\slshape vari}}. }

 

\subsection{\textcolor{Chapter }{CotangentSheaf (for IsToricVariety)}}
\logpage{[ 3, 4, 24 ]}\nobreak
\hyperdef{L}{X875E6958856A7833}{}
{\noindent\textcolor{FuncColor}{$\triangleright$\enspace\texttt{CotangentSheaf({\mdseries\slshape vari})\index{CotangentSheaf@\texttt{CotangentSheaf}!for IsToricVariety}
\label{CotangentSheaf:for IsToricVariety}
}\hfill{\scriptsize (attribute)}}\\
\textbf{\indent Returns:\ }
a f.p. graded \mbox{\texttt{\mdseries\slshape S}}-module 



 This method returns a f. p. graded \mbox{\texttt{\mdseries\slshape S}}-module (\mbox{\texttt{\mdseries\slshape S}} being the Cox ring of the variety), such that the sheafification of this
module is the cotangent sheaf of \mbox{\texttt{\mdseries\slshape vari}}. }

 

\subsection{\textcolor{Chapter }{EulerCharacteristic (for IsToricVariety)}}
\logpage{[ 3, 4, 25 ]}\nobreak
\hyperdef{L}{X7AEC5F2B7FBF4B4B}{}
{\noindent\textcolor{FuncColor}{$\triangleright$\enspace\texttt{EulerCharacteristic({\mdseries\slshape vari})\index{EulerCharacteristic@\texttt{EulerCharacteristic}!for IsToricVariety}
\label{EulerCharacteristic:for IsToricVariety}
}\hfill{\scriptsize (attribute)}}\\
\textbf{\indent Returns:\ }
a non-negative integer 



 This method computes the Euler characteristic of \mbox{\texttt{\mdseries\slshape vari}}. }

 }

 
\section{\textcolor{Chapter }{Methods}}\label{Chapter_Toric_Varieties_Section_Methods}
\logpage{[ 3, 5, 0 ]}
\hyperdef{L}{X8606FDCE878850EF}{}
{
  

\subsection{\textcolor{Chapter }{UnderlyingSheaf (for IsToricVariety)}}
\logpage{[ 3, 5, 1 ]}\nobreak
\hyperdef{L}{X7B4B8A9A876591AA}{}
{\noindent\textcolor{FuncColor}{$\triangleright$\enspace\texttt{UnderlyingSheaf({\mdseries\slshape vari})\index{UnderlyingSheaf@\texttt{UnderlyingSheaf}!for IsToricVariety}
\label{UnderlyingSheaf:for IsToricVariety}
}\hfill{\scriptsize (operation)}}\\
\textbf{\indent Returns:\ }
a sheaf 



 The method returns the underlying sheaf of the variety \mbox{\texttt{\mdseries\slshape vari}}. }

 

\subsection{\textcolor{Chapter }{CoordinateRingOfTorus (for IsToricVariety, IsList)}}
\logpage{[ 3, 5, 2 ]}\nobreak
\hyperdef{L}{X7864ED227FC67EAF}{}
{\noindent\textcolor{FuncColor}{$\triangleright$\enspace\texttt{CoordinateRingOfTorus({\mdseries\slshape vari, vars})\index{CoordinateRingOfTorus@\texttt{CoordinateRingOfTorus}!for IsToricVariety, IsList}
\label{CoordinateRingOfTorus:for IsToricVariety, IsList}
}\hfill{\scriptsize (operation)}}\\
\textbf{\indent Returns:\ }
a ring 



 Computes the coordinate ring of the torus of the variety \mbox{\texttt{\mdseries\slshape vari}} with the variables \mbox{\texttt{\mdseries\slshape vars}}. The argument \mbox{\texttt{\mdseries\slshape vars}} need to be a list of strings with length dimension or two times dimension. }

 

\subsection{\textcolor{Chapter }{\texttt{\symbol{92}}* (for IsToricVariety, IsToricVariety)}}
\logpage{[ 3, 5, 3 ]}\nobreak
\hyperdef{L}{X800F95657C3CCBC2}{}
{\noindent\textcolor{FuncColor}{$\triangleright$\enspace\texttt{\texttt{\symbol{92}}*({\mdseries\slshape vari1, vari2})\index{\texttt{\symbol{92}}*@\texttt{\texttt{\symbol{92}}*}!for IsToricVariety, IsToricVariety}
\label{bSlash*:for IsToricVariety, IsToricVariety}
}\hfill{\scriptsize (operation)}}\\
\textbf{\indent Returns:\ }
a variety 



 Computes the categorial product of the varieties \mbox{\texttt{\mdseries\slshape vari1}} and \mbox{\texttt{\mdseries\slshape vari2}}. }

 

\subsection{\textcolor{Chapter }{CharacterToRationalFunction (for IsHomalgElement, IsToricVariety)}}
\logpage{[ 3, 5, 4 ]}\nobreak
\hyperdef{L}{X7EB6407381DA770B}{}
{\noindent\textcolor{FuncColor}{$\triangleright$\enspace\texttt{CharacterToRationalFunction({\mdseries\slshape elem, vari})\index{CharacterToRationalFunction@\texttt{CharacterToRationalFunction}!for IsHomalgElement, IsToricVariety}
\label{CharacterToRationalFunction:for IsHomalgElement, IsToricVariety}
}\hfill{\scriptsize (operation)}}\\
\textbf{\indent Returns:\ }
a homalg element 



 Computes the rational function corresponding to the character grid element \mbox{\texttt{\mdseries\slshape elem}} or to the list of integers \mbox{\texttt{\mdseries\slshape elem}}. This computation needs to know the coordinate ring of the torus of the
variety \mbox{\texttt{\mdseries\slshape vari}}. By default this ring is introduced with variables \mbox{\texttt{\mdseries\slshape x1}} to \mbox{\texttt{\mdseries\slshape xn}} where \mbox{\texttt{\mdseries\slshape n}} is the dimension of the variety. If different variables should be used, then \mbox{\texttt{\mdseries\slshape CoordinateRingOfTorus}} has to be set accordingly before calling this method. }

 

\subsection{\textcolor{Chapter }{CoxRing (for IsToricVariety, IsList)}}
\logpage{[ 3, 5, 5 ]}\nobreak
\hyperdef{L}{X82AB2367795C5B53}{}
{\noindent\textcolor{FuncColor}{$\triangleright$\enspace\texttt{CoxRing({\mdseries\slshape vari, vars})\index{CoxRing@\texttt{CoxRing}!for IsToricVariety, IsList}
\label{CoxRing:for IsToricVariety, IsList}
}\hfill{\scriptsize (operation)}}\\
\textbf{\indent Returns:\ }
a ring 



 Computes the Cox ring of the variety \mbox{\texttt{\mdseries\slshape vari}}. \mbox{\texttt{\mdseries\slshape vars}} needs to be a string. We allow for two different formats. Either, it is a
string which does not contain ",". Then this string will be index and the
resulting strings are then used as names for the variables of the Cox ring.
Alternatively, one can also use a string containing ",". In this case, a ","
is considered as separator and one can provide individual names for all
variables of the Cox ring. }

 

\subsection{\textcolor{Chapter }{WeilDivisorsOfVariety (for IsToricVariety)}}
\logpage{[ 3, 5, 6 ]}\nobreak
\hyperdef{L}{X81304E6685492462}{}
{\noindent\textcolor{FuncColor}{$\triangleright$\enspace\texttt{WeilDivisorsOfVariety({\mdseries\slshape vari})\index{WeilDivisorsOfVariety@\texttt{WeilDivisorsOfVariety}!for IsToricVariety}
\label{WeilDivisorsOfVariety:for IsToricVariety}
}\hfill{\scriptsize (operation)}}\\
\textbf{\indent Returns:\ }
a list 



 Returns a list of the currently defined Divisors of the toric variety. }

 

\subsection{\textcolor{Chapter }{Fan (for IsToricVariety)}}
\logpage{[ 3, 5, 7 ]}\nobreak
\hyperdef{L}{X7A073EC97B7B2F58}{}
{\noindent\textcolor{FuncColor}{$\triangleright$\enspace\texttt{Fan({\mdseries\slshape vari})\index{Fan@\texttt{Fan}!for IsToricVariety}
\label{Fan:for IsToricVariety}
}\hfill{\scriptsize (operation)}}\\
\textbf{\indent Returns:\ }
a fan 



 Returns the fan of the variety \mbox{\texttt{\mdseries\slshape vari}}. This is a rename for FanOfVariety. }

 

\subsection{\textcolor{Chapter }{Factors (for IsToricVariety)}}
\logpage{[ 3, 5, 8 ]}\nobreak
\hyperdef{L}{X7C48FD287E0B4F68}{}
{\noindent\textcolor{FuncColor}{$\triangleright$\enspace\texttt{Factors({\mdseries\slshape vari})\index{Factors@\texttt{Factors}!for IsToricVariety}
\label{Factors:for IsToricVariety}
}\hfill{\scriptsize (operation)}}\\


 

 }

 

\subsection{\textcolor{Chapter }{BlowUpOnIthMinimalTorusOrbit (for IsToricVariety, IsInt)}}
\logpage{[ 3, 5, 9 ]}\nobreak
\hyperdef{L}{X7F9BA4A5850A7D3C}{}
{\noindent\textcolor{FuncColor}{$\triangleright$\enspace\texttt{BlowUpOnIthMinimalTorusOrbit({\mdseries\slshape vari, p})\index{BlowUpOnIthMinimalTorusOrbit@\texttt{BlowUpOnIthMinimalTorusOrbit}!for IsToricVariety, IsInt}
\label{BlowUpOnIthMinimalTorusOrbit:for IsToricVariety, IsInt}
}\hfill{\scriptsize (operation)}}\\


 

 }

 

\subsection{\textcolor{Chapter }{ZariskiCotangentSheafViaEulerSequence}}
\logpage{[ 3, 5, 10 ]}\nobreak
\hyperdef{L}{X807354E58223DA87}{}
{\noindent\textcolor{FuncColor}{$\triangleright$\enspace\texttt{ZariskiCotangentSheafViaEulerSequence({\mdseries\slshape arg})\index{ZariskiCotangentSheafViaEulerSequence@\texttt{Zariski}\-\texttt{Cotangent}\-\texttt{Sheaf}\-\texttt{Via}\-\texttt{Euler}\-\texttt{Sequence}}
\label{ZariskiCotangentSheafViaEulerSequence}
}\hfill{\scriptsize (function)}}\\


 

 }

 

\subsection{\textcolor{Chapter }{ZariskiCotangentSheafViaPoincareResidueMap}}
\logpage{[ 3, 5, 11 ]}\nobreak
\hyperdef{L}{X7EF0140085B37534}{}
{\noindent\textcolor{FuncColor}{$\triangleright$\enspace\texttt{ZariskiCotangentSheafViaPoincareResidueMap({\mdseries\slshape arg})\index{ZariskiCotangentSheafViaPoincareResidueMap@\texttt{Zariski}\-\texttt{Cotangent}\-\texttt{Sheaf}\-\texttt{Via}\-\texttt{Poincare}\-\texttt{ResidueMap}}
\label{ZariskiCotangentSheafViaPoincareResidueMap}
}\hfill{\scriptsize (function)}}\\


 

 }

 

\subsection{\textcolor{Chapter }{ithBettiNumber (for IsToricVariety, IsInt)}}
\logpage{[ 3, 5, 12 ]}\nobreak
\hyperdef{L}{X8175E6A879BD0FFE}{}
{\noindent\textcolor{FuncColor}{$\triangleright$\enspace\texttt{ithBettiNumber({\mdseries\slshape vari, p})\index{ithBettiNumber@\texttt{ithBettiNumber}!for IsToricVariety, IsInt}
\label{ithBettiNumber:for IsToricVariety, IsInt}
}\hfill{\scriptsize (operation)}}\\


 

 }

 

\subsection{\textcolor{Chapter }{NrOfqRationalPoints (for IsToricVariety, IsInt)}}
\logpage{[ 3, 5, 13 ]}\nobreak
\hyperdef{L}{X87AE434F7CC3DA0A}{}
{\noindent\textcolor{FuncColor}{$\triangleright$\enspace\texttt{NrOfqRationalPoints({\mdseries\slshape vari, p})\index{NrOfqRationalPoints@\texttt{NrOfqRationalPoints}!for IsToricVariety, IsInt}
\label{NrOfqRationalPoints:for IsToricVariety, IsInt}
}\hfill{\scriptsize (operation)}}\\


 

 }

 }

 
\section{\textcolor{Chapter }{Constructors}}\label{Chapter_Toric_Varieties_Section_Constructors}
\logpage{[ 3, 6, 0 ]}
\hyperdef{L}{X86EC0F0A78ECBC10}{}
{
  

\subsection{\textcolor{Chapter }{ToricVariety (for IsToricVariety)}}
\logpage{[ 3, 6, 1 ]}\nobreak
\hyperdef{L}{X7D8E4D46802FED13}{}
{\noindent\textcolor{FuncColor}{$\triangleright$\enspace\texttt{ToricVariety({\mdseries\slshape vari})\index{ToricVariety@\texttt{ToricVariety}!for IsToricVariety}
\label{ToricVariety:for IsToricVariety}
}\hfill{\scriptsize (operation)}}\\


 

 }

 

\subsection{\textcolor{Chapter }{ToricVariety (for IsList)}}
\logpage{[ 3, 6, 2 ]}\nobreak
\hyperdef{L}{X8202F293835E28AC}{}
{\noindent\textcolor{FuncColor}{$\triangleright$\enspace\texttt{ToricVariety({\mdseries\slshape vari})\index{ToricVariety@\texttt{ToricVariety}!for IsList}
\label{ToricVariety:for IsList}
}\hfill{\scriptsize (operation)}}\\


 

 }

 

\subsection{\textcolor{Chapter }{ToricVariety (for IsConvexObject)}}
\logpage{[ 3, 6, 3 ]}\nobreak
\hyperdef{L}{X84BB2EF387E22171}{}
{\noindent\textcolor{FuncColor}{$\triangleright$\enspace\texttt{ToricVariety({\mdseries\slshape conv})\index{ToricVariety@\texttt{ToricVariety}!for IsConvexObject}
\label{ToricVariety:for IsConvexObject}
}\hfill{\scriptsize (operation)}}\\
\textbf{\indent Returns:\ }
a variety 



 Creates a toric variety out of the convex object \mbox{\texttt{\mdseries\slshape conv}}. }

 

\subsection{\textcolor{Chapter }{ToricVariety (for IsList, IsList, IsList)}}
\logpage{[ 3, 6, 4 ]}\nobreak
\hyperdef{L}{X87E98D4E82571595}{}
{\noindent\textcolor{FuncColor}{$\triangleright$\enspace\texttt{ToricVariety({\mdseries\slshape rays, cones, degree{\textunderscore}list})\index{ToricVariety@\texttt{ToricVariety}!for IsList, IsList, IsList}
\label{ToricVariety:for IsList, IsList, IsList}
}\hfill{\scriptsize (operation)}}\\
\textbf{\indent Returns:\ }
a variety 



 Creates a toric variety from a list \mbox{\texttt{\mdseries\slshape rays}} of ray generators and cones \mbox{\texttt{\mdseries\slshape cones}}. Beyond the functionality of the other methods, this constructor allows to
assign specific gradings to the homogeneous variables of the Cox ring. With
respect to the order in which the rays appear in the list \mbox{\texttt{\mdseries\slshape rays}}, we assign gradings as provided by the third argument \mbox{\texttt{\mdseries\slshape degree{\textunderscore}list}} . The latter is a list of integers. }

 

\subsection{\textcolor{Chapter }{ToricVariety (for IsList, IsList, IsList, IsList)}}
\logpage{[ 3, 6, 5 ]}\nobreak
\hyperdef{L}{X81C81F337B8C27D9}{}
{\noindent\textcolor{FuncColor}{$\triangleright$\enspace\texttt{ToricVariety({\mdseries\slshape rays, cones, degree{\textunderscore}list, var{\textunderscore}list})\index{ToricVariety@\texttt{ToricVariety}!for IsList, IsList, IsList, IsList}
\label{ToricVariety:for IsList, IsList, IsList, IsList}
}\hfill{\scriptsize (operation)}}\\
\textbf{\indent Returns:\ }
a variety 



 Creates a toric variety from a list \mbox{\texttt{\mdseries\slshape rays}} of ray generators and cones \mbox{\texttt{\mdseries\slshape cones}}. Beyond the functionality of the other methods, this constructor allows to
assign specific gradings and homogeneous variable names to the ray generators
of this toric variety. With respect to the order in which the rays appear in
the list \mbox{\texttt{\mdseries\slshape rays}}, we assign gradings and variable names as provided by the third and fourth
argument. These are the list of gradings \mbox{\texttt{\mdseries\slshape degree{\textunderscore}list}} and the list of variables names \mbox{\texttt{\mdseries\slshape var{\textunderscore}list}}. The former is a list of integers and the latter a list of strings. }

 

\subsection{\textcolor{Chapter }{ToricVarietiesFromGrading (for IsList)}}
\logpage{[ 3, 6, 6 ]}\nobreak
\hyperdef{L}{X837CE639825FEC9B}{}
{\noindent\textcolor{FuncColor}{$\triangleright$\enspace\texttt{ToricVarietiesFromGrading({\mdseries\slshape a, list, of, lists, of, integers})\index{ToricVarietiesFromGrading@\texttt{ToricVarietiesFromGrading}!for IsList}
\label{ToricVarietiesFromGrading:for IsList}
}\hfill{\scriptsize (operation)}}\\
\textbf{\indent Returns:\ }
a list of toric varieties 



 Given a $\mathbb{Z}^n$-grading of a polynomial ring, this method computes all toric varieties, which
are normal and have no-torus factor and whose Cox ring obeys the given $\mathbb{Z}^n$-grading. }

 

\subsection{\textcolor{Chapter }{ToricVarietyFromGrading (for IsList)}}
\logpage{[ 3, 6, 7 ]}\nobreak
\hyperdef{L}{X7965EE327B3EB0CF}{}
{\noindent\textcolor{FuncColor}{$\triangleright$\enspace\texttt{ToricVarietyFromGrading({\mdseries\slshape a, list, of, lists, of, integers})\index{ToricVarietyFromGrading@\texttt{ToricVarietyFromGrading}!for IsList}
\label{ToricVarietyFromGrading:for IsList}
}\hfill{\scriptsize (operation)}}\\
\textbf{\indent Returns:\ }
a toric variety 



 Given a $\mathbb{Z}^n$-grading of a polynomial ring, this method computes a toric variety, which is
normal and has no-torus factor and whose Cox ring obeys the given $\mathbb{Z}^n$-grading. }

 }

 }

   
\chapter{\textcolor{Chapter }{Toric subvarieties}}\label{Chapter_Toric_subvarieties}
\logpage{[ 4, 0, 0 ]}
\hyperdef{L}{X84370283823C138C}{}
{
  

 
\section{\textcolor{Chapter }{The GAP category}}\label{Chapter_Toric_subvarieties_Section_The_GAP_category}
\logpage{[ 4, 1, 0 ]}
\hyperdef{L}{X7819115082F3E2A6}{}
{
  

\subsection{\textcolor{Chapter }{IsToricSubvariety (for IsToricVariety)}}
\logpage{[ 4, 1, 1 ]}\nobreak
\hyperdef{L}{X87E213557E5F2716}{}
{\noindent\textcolor{FuncColor}{$\triangleright$\enspace\texttt{IsToricSubvariety({\mdseries\slshape M})\index{IsToricSubvariety@\texttt{IsToricSubvariety}!for IsToricVariety}
\label{IsToricSubvariety:for IsToricVariety}
}\hfill{\scriptsize (filter)}}\\
\textbf{\indent Returns:\ }
true or false 



 The \mbox{\texttt{\mdseries\slshape GAP}} category of a toric subvariety. Every toric subvariety is a toric variety, so
every method applicable to toric varieties is also applicable to toric
subvarieties. }

 }

 
\section{\textcolor{Chapter }{Properties}}\label{Chapter_Toric_subvarieties_Section_Properties}
\logpage{[ 4, 2, 0 ]}
\hyperdef{L}{X871597447BB998A1}{}
{
  

\subsection{\textcolor{Chapter }{IsClosedSubvariety (for IsToricSubvariety)}}
\logpage{[ 4, 2, 1 ]}\nobreak
\hyperdef{L}{X87BF07957A6F7DFC}{}
{\noindent\textcolor{FuncColor}{$\triangleright$\enspace\texttt{IsClosedSubvariety({\mdseries\slshape vari})\index{IsClosedSubvariety@\texttt{IsClosedSubvariety}!for IsToricSubvariety}
\label{IsClosedSubvariety:for IsToricSubvariety}
}\hfill{\scriptsize (property)}}\\
\textbf{\indent Returns:\ }
true or false 



 Checks if the subvariety \mbox{\texttt{\mdseries\slshape vari}} is a closed subset of its ambient variety. }

 

\subsection{\textcolor{Chapter }{IsOpen (for IsToricSubvariety)}}
\logpage{[ 4, 2, 2 ]}\nobreak
\hyperdef{L}{X7BACBD04825801D6}{}
{\noindent\textcolor{FuncColor}{$\triangleright$\enspace\texttt{IsOpen({\mdseries\slshape vari})\index{IsOpen@\texttt{IsOpen}!for IsToricSubvariety}
\label{IsOpen:for IsToricSubvariety}
}\hfill{\scriptsize (property)}}\\
\textbf{\indent Returns:\ }
true or false 



 Checks if a subvariety is a closed subset. }

 

\subsection{\textcolor{Chapter }{IsWholeVariety (for IsToricSubvariety)}}
\logpage{[ 4, 2, 3 ]}\nobreak
\hyperdef{L}{X84653ABF7DEB60D4}{}
{\noindent\textcolor{FuncColor}{$\triangleright$\enspace\texttt{IsWholeVariety({\mdseries\slshape vari})\index{IsWholeVariety@\texttt{IsWholeVariety}!for IsToricSubvariety}
\label{IsWholeVariety:for IsToricSubvariety}
}\hfill{\scriptsize (property)}}\\
\textbf{\indent Returns:\ }
true or false 



 Returns true if the subvariety \mbox{\texttt{\mdseries\slshape vari}} is the whole variety. }

 }

 
\section{\textcolor{Chapter }{Attributes}}\label{Chapter_Toric_subvarieties_Section_Attributes}
\logpage{[ 4, 3, 0 ]}
\hyperdef{L}{X7C701DBF7BAE649A}{}
{
  

\subsection{\textcolor{Chapter }{UnderlyingToricVariety (for IsToricSubvariety)}}
\logpage{[ 4, 3, 1 ]}\nobreak
\hyperdef{L}{X7D34E83578D38C30}{}
{\noindent\textcolor{FuncColor}{$\triangleright$\enspace\texttt{UnderlyingToricVariety({\mdseries\slshape vari})\index{UnderlyingToricVariety@\texttt{UnderlyingToricVariety}!for IsToricSubvariety}
\label{UnderlyingToricVariety:for IsToricSubvariety}
}\hfill{\scriptsize (attribute)}}\\
\textbf{\indent Returns:\ }
a variety 



 Returns the toric variety which is represented by \mbox{\texttt{\mdseries\slshape vari}}. This method implements the forgetful functor subvarieties -{\textgreater}
varieties. }

 

\subsection{\textcolor{Chapter }{InclusionMorphism (for IsToricSubvariety)}}
\logpage{[ 4, 3, 2 ]}\nobreak
\hyperdef{L}{X867EB68F85A046ED}{}
{\noindent\textcolor{FuncColor}{$\triangleright$\enspace\texttt{InclusionMorphism({\mdseries\slshape vari})\index{InclusionMorphism@\texttt{InclusionMorphism}!for IsToricSubvariety}
\label{InclusionMorphism:for IsToricSubvariety}
}\hfill{\scriptsize (attribute)}}\\
\textbf{\indent Returns:\ }
a morphism 



 If the variety \mbox{\texttt{\mdseries\slshape vari}} is an open subvariety, this method returns the inclusion morphism in its
ambient variety. If not, it will fail. }

 

\subsection{\textcolor{Chapter }{AmbientToricVariety (for IsToricSubvariety)}}
\logpage{[ 4, 3, 3 ]}\nobreak
\hyperdef{L}{X81765D817B1DEF4E}{}
{\noindent\textcolor{FuncColor}{$\triangleright$\enspace\texttt{AmbientToricVariety({\mdseries\slshape vari})\index{AmbientToricVariety@\texttt{AmbientToricVariety}!for IsToricSubvariety}
\label{AmbientToricVariety:for IsToricSubvariety}
}\hfill{\scriptsize (attribute)}}\\
\textbf{\indent Returns:\ }
a variety 



 Returns the ambient toric variety of the subvariety \mbox{\texttt{\mdseries\slshape vari}} }

 }

 
\section{\textcolor{Chapter }{Methods}}\label{Chapter_Toric_subvarieties_Section_Methods}
\logpage{[ 4, 4, 0 ]}
\hyperdef{L}{X8606FDCE878850EF}{}
{
  

\subsection{\textcolor{Chapter }{ClosureOfTorusOrbitOfCone (for IsToricVariety, IsCone)}}
\logpage{[ 4, 4, 1 ]}\nobreak
\hyperdef{L}{X7F23E4678532E242}{}
{\noindent\textcolor{FuncColor}{$\triangleright$\enspace\texttt{ClosureOfTorusOrbitOfCone({\mdseries\slshape vari, cone})\index{ClosureOfTorusOrbitOfCone@\texttt{ClosureOfTorusOrbitOfCone}!for IsToricVariety, IsCone}
\label{ClosureOfTorusOrbitOfCone:for IsToricVariety, IsCone}
}\hfill{\scriptsize (operation)}}\\
\textbf{\indent Returns:\ }
a subvariety 



 The method returns the closure of the orbit of the torus contained in \mbox{\texttt{\mdseries\slshape vari}} which corresponds to the cone \mbox{\texttt{\mdseries\slshape cone}} as a closed subvariety of \mbox{\texttt{\mdseries\slshape vari}}. }

 }

 
\section{\textcolor{Chapter }{Constructors}}\label{Chapter_Toric_subvarieties_Section_Constructors}
\logpage{[ 4, 5, 0 ]}
\hyperdef{L}{X86EC0F0A78ECBC10}{}
{
  

\subsection{\textcolor{Chapter }{ToricSubvariety (for IsToricVariety, IsToricVariety)}}
\logpage{[ 4, 5, 1 ]}\nobreak
\hyperdef{L}{X8554CD3F7EB77AC8}{}
{\noindent\textcolor{FuncColor}{$\triangleright$\enspace\texttt{ToricSubvariety({\mdseries\slshape vari, ambvari})\index{ToricSubvariety@\texttt{ToricSubvariety}!for IsToricVariety, IsToricVariety}
\label{ToricSubvariety:for IsToricVariety, IsToricVariety}
}\hfill{\scriptsize (operation)}}\\
\textbf{\indent Returns:\ }
a subvariety 



 The method returns the closure of the orbit of the torus contained in \mbox{\texttt{\mdseries\slshape vari}} which corresponds to the cone \mbox{\texttt{\mdseries\slshape cone}} as a closed subvariety of \mbox{\texttt{\mdseries\slshape vari}}. }

 }

 }

   
\chapter{\textcolor{Chapter }{Affine toric varieties}}\label{Chapter_Affine_toric_varieties}
\logpage{[ 5, 0, 0 ]}
\hyperdef{L}{X82F418F483E4D0D6}{}
{
  

 
\section{\textcolor{Chapter }{Affine toric varieties: Examples}}\label{Chapter_Affine_toric_varieties_Section_Affine_toric_varieties_Examples}
\logpage{[ 5, 1, 0 ]}
\hyperdef{L}{X8068F91F7C001DE7}{}
{
  

 
\subsection{\textcolor{Chapter }{Affine space}}\label{Chapter_Affine_toric_varieties_Section_Affine_toric_varieties_Examples_Subsection_Affine_space}
\logpage{[ 5, 1, 1 ]}
\hyperdef{L}{X782DF75D8761D85B}{}
{
  
\begin{Verbatim}[commandchars=!@B,fontsize=\small,frame=single,label=Example]
  !gapprompt@gap>B !gapinput@F := Fan( [[1,0,0],[0,1,0],[0,0,1]], [[1,2,3]] );B
  <A fan in |R^3>
  !gapprompt@gap>B !gapinput@C3 := ToricVariety( F );B
  <A toric variety of dimension 3>
  !gapprompt@gap>B !gapinput@IsAffine( C3 );B
  true
  !gapprompt@gap>B !gapinput@Dimension( C3 );B
  3
\end{Verbatim}
 More conveniently, we can build affine toric varieties from a cone: 
\begin{Verbatim}[commandchars=!@B,fontsize=\small,frame=single,label=Example]
  !gapprompt@gap>B !gapinput@IsAffine( ProjectiveSpace( 1 ) );B
  false
  !gapprompt@gap>B !gapinput@C:=Cone( [[1,0,0],[0,1,0],[0,0,1]] );B
  <A cone in |R^3>
  !gapprompt@gap>B !gapinput@C3:=ToricVariety(C);B
  <An affine normal toric variety of dimension 3>
  !gapprompt@gap>B !gapinput@Dimension(C3);B
  3
  !gapprompt@gap>B !gapinput@IsSimplicial( C3 );B
  true
  !gapprompt@gap>B !gapinput@IsOrbifold(C3);B
  true
  !gapprompt@gap>B !gapinput@IsSmooth(C3);B
  true
  !gapprompt@gap>B !gapinput@IsProjective( C3 );B
  false
  !gapprompt@gap>B !gapinput@DimensionOfTorusfactor( C3 );B
  0
  !gapprompt@gap>B !gapinput@CoordinateRingOfTorus(C3,"x");B
  Q[x1,x1_,x2,x2_,x3,x3_]/( x1*x1_-1, x2*x2_-1, x3*x3_-1 )
  !gapprompt@gap>B !gapinput@CoordinateRing(C3,"x");B
  Q[x_1,x_2,x_3]
  !gapprompt@gap>B !gapinput@ListOfVariablesOfCoordinateRing( C3 );B
  [ "x_1", "x_2", "x_3" ]
  !gapprompt@gap>B !gapinput@MorphismFromCoordinateRingToCoordinateRingOfTorus( C3 );B
  <A monomorphism of rings>
  !gapprompt@gap>B !gapinput@C3;B
  <An affine normal smooth toric variety of dimension 3>
  !gapprompt@gap>B !gapinput@StructureDescription( C3 );B
  "|A^3"
  !gapprompt@gap>B !gapinput@ConeOfVariety( C3 );B
  <A smooth pointed simplicial cone in |R^3 with 3 ray generators>
  !gapprompt@gap>B !gapinput@Cone( C3 );B
  <A smooth pointed simplicial cone in |R^3 with 3 ray generators>
  !gapprompt@gap>B !gapinput@IrrelevantIdeal( C3 );B
  <A graded principal torsion-free (left) ideal given by a cyclic generator>
  !gapprompt@gap>B !gapinput@CartierTorusInvariantDivisorGroup( C3 );B
  <A free left submodule given by 3 generators>
\end{Verbatim}
 
\begin{Verbatim}[commandchars=!@B,fontsize=\small,frame=single,label=Example]
  !gapprompt@gap>B !gapinput@v:=Cone( [[1,0,0],[0,1,0]] );B
  <A cone in |R^3>
  !gapprompt@gap>B !gapinput@v:=ToricVariety(v);B
  <An affine normal toric variety of dimension 3>
  !gapprompt@gap>B !gapinput@DimensionOfTorusfactor( v );B
  1
  !gapprompt@gap>B !gapinput@CartierTorusInvariantDivisorGroup( v );B
  <A free left submodule given by 3 generators>
  !gapprompt@gap>B !gapinput@ConeOfVariety( v );B
  <A pointed cone in |R^3 of dimension 2 with 2 ray generators>
  !gapprompt@gap>B !gapinput@Cone( v );B
  <A pointed cone in |R^3 of dimension 2 with 2 ray generators>
\end{Verbatim}
 
\begin{Verbatim}[commandchars=!@B,fontsize=\small,frame=single,label=Example]
  !gapprompt@gap>B !gapinput@v2:=Cone( [[1,1],[-1,1]] );B
  <A cone in |R^2>
  !gapprompt@gap>B !gapinput@v2:=ToricVariety(v2);B
  <An affine normal toric variety of dimension 2>
  !gapprompt@gap>B !gapinput@IsSmooth( v2 );B
  false
  !gapprompt@gap>B !gapinput@Display( v2 );B
  An affine normal non smooth toric variety of dimension 2.
  !gapprompt@gap>B !gapinput@ConeOfVariety( v * v2 );B
  <A pointed cone in |R^5>
\end{Verbatim}
 

 }

 }

 
\section{\textcolor{Chapter }{The GAP category}}\label{Chapter_Affine_toric_varieties_Section_The_GAP_category}
\logpage{[ 5, 2, 0 ]}
\hyperdef{L}{X7819115082F3E2A6}{}
{
  

\subsection{\textcolor{Chapter }{IsAffineToricVariety (for IsToricVariety)}}
\logpage{[ 5, 2, 1 ]}\nobreak
\hyperdef{L}{X79DAFC45797A1DAD}{}
{\noindent\textcolor{FuncColor}{$\triangleright$\enspace\texttt{IsAffineToricVariety({\mdseries\slshape M})\index{IsAffineToricVariety@\texttt{IsAffineToricVariety}!for IsToricVariety}
\label{IsAffineToricVariety:for IsToricVariety}
}\hfill{\scriptsize (filter)}}\\
\textbf{\indent Returns:\ }
true or false 



 The \mbox{\texttt{\mdseries\slshape GAP}} category of an affine toric variety. All affine toric varieties are toric
varieties, so everything applicable to toric varieties is applicable to affine
toric varieties. }

 }

 
\section{\textcolor{Chapter }{Attributes}}\label{Chapter_Affine_toric_varieties_Section_Attributes}
\logpage{[ 5, 3, 0 ]}
\hyperdef{L}{X7C701DBF7BAE649A}{}
{
  

\subsection{\textcolor{Chapter }{CoordinateRing (for IsAffineToricVariety)}}
\logpage{[ 5, 3, 1 ]}\nobreak
\hyperdef{L}{X7E22127584BB20FA}{}
{\noindent\textcolor{FuncColor}{$\triangleright$\enspace\texttt{CoordinateRing({\mdseries\slshape vari})\index{CoordinateRing@\texttt{CoordinateRing}!for IsAffineToricVariety}
\label{CoordinateRing:for IsAffineToricVariety}
}\hfill{\scriptsize (attribute)}}\\
\textbf{\indent Returns:\ }
a ring 



 Returns the coordinate ring of the affine toric variety \mbox{\texttt{\mdseries\slshape vari}}. }

 

\subsection{\textcolor{Chapter }{ListOfVariablesOfCoordinateRing (for IsAffineToricVariety)}}
\logpage{[ 5, 3, 2 ]}\nobreak
\hyperdef{L}{X8205D8A97B0C130B}{}
{\noindent\textcolor{FuncColor}{$\triangleright$\enspace\texttt{ListOfVariablesOfCoordinateRing({\mdseries\slshape vari})\index{ListOfVariablesOfCoordinateRing@\texttt{ListOfVariablesOfCoordinateRing}!for IsAffineToricVariety}
\label{ListOfVariablesOfCoordinateRing:for IsAffineToricVariety}
}\hfill{\scriptsize (attribute)}}\\
\textbf{\indent Returns:\ }
a list 



 Returns a list containing the variables of the CoordinateRing of the variety \mbox{\texttt{\mdseries\slshape vari}}. }

 

\subsection{\textcolor{Chapter }{MorphismFromCoordinateRingToCoordinateRingOfTorus (for IsToricVariety)}}
\logpage{[ 5, 3, 3 ]}\nobreak
\hyperdef{L}{X858127C1879C9539}{}
{\noindent\textcolor{FuncColor}{$\triangleright$\enspace\texttt{MorphismFromCoordinateRingToCoordinateRingOfTorus({\mdseries\slshape vari})\index{MorphismFromCoordinateRingToCoordinateRingOfTorus@\texttt{Morphism}\-\texttt{From}\-\texttt{Coordinate}\-\texttt{Ring}\-\texttt{To}\-\texttt{Coordinate}\-\texttt{Ring}\-\texttt{Of}\-\texttt{Torus}!for IsToricVariety}
\label{MorphismFromCoordinateRingToCoordinateRingOfTorus:for IsToricVariety}
}\hfill{\scriptsize (attribute)}}\\
\textbf{\indent Returns:\ }
a morphism 



 Returns the morphism between the coordinate ring of the variety \mbox{\texttt{\mdseries\slshape vari}} and the coordinate ring of its torus. This defines the embedding of the torus
in the variety. }

 

\subsection{\textcolor{Chapter }{ConeOfVariety (for IsToricVariety)}}
\logpage{[ 5, 3, 4 ]}\nobreak
\hyperdef{L}{X7AFEEEA184161182}{}
{\noindent\textcolor{FuncColor}{$\triangleright$\enspace\texttt{ConeOfVariety({\mdseries\slshape vari})\index{ConeOfVariety@\texttt{ConeOfVariety}!for IsToricVariety}
\label{ConeOfVariety:for IsToricVariety}
}\hfill{\scriptsize (attribute)}}\\
\textbf{\indent Returns:\ }
a cone 



 Returns the cone of the affine toric variety \mbox{\texttt{\mdseries\slshape vari}}. }

 }

 
\section{\textcolor{Chapter }{Methods}}\label{Chapter_Affine_toric_varieties_Section_Methods}
\logpage{[ 5, 4, 0 ]}
\hyperdef{L}{X8606FDCE878850EF}{}
{
  

\subsection{\textcolor{Chapter }{CoordinateRing (for IsToricVariety, IsList)}}
\logpage{[ 5, 4, 1 ]}\nobreak
\hyperdef{L}{X87A1BD8578B73E05}{}
{\noindent\textcolor{FuncColor}{$\triangleright$\enspace\texttt{CoordinateRing({\mdseries\slshape vari, indet})\index{CoordinateRing@\texttt{CoordinateRing}!for IsToricVariety, IsList}
\label{CoordinateRing:for IsToricVariety, IsList}
}\hfill{\scriptsize (operation)}}\\
\textbf{\indent Returns:\ }
a ring 



 Computes the coordinate ring of the affine toric variety \mbox{\texttt{\mdseries\slshape vari}} with indeterminates \mbox{\texttt{\mdseries\slshape indet}}. }

 

\subsection{\textcolor{Chapter }{Cone (for IsToricVariety)}}
\logpage{[ 5, 4, 2 ]}\nobreak
\hyperdef{L}{X83AB831683AF0F27}{}
{\noindent\textcolor{FuncColor}{$\triangleright$\enspace\texttt{Cone({\mdseries\slshape vari})\index{Cone@\texttt{Cone}!for IsToricVariety}
\label{Cone:for IsToricVariety}
}\hfill{\scriptsize (operation)}}\\
\textbf{\indent Returns:\ }
a cone 



 Returns the cone of the variety \mbox{\texttt{\mdseries\slshape vari}}. Another name for ConeOfVariety for compatibility and shortness. }

 }

 
\section{\textcolor{Chapter }{Constructors}}\label{Chapter_Affine_toric_varieties_Section_Constructors}
\logpage{[ 5, 5, 0 ]}
\hyperdef{L}{X86EC0F0A78ECBC10}{}
{
  The constructors are the same as for toric varieties. Calling them with a cone
will result in an affine variety. }

 }

   
\chapter{\textcolor{Chapter }{Projective toric varieties}}\label{Chapter_Projective_toric_varieties}
\logpage{[ 6, 0, 0 ]}
\hyperdef{L}{X7EEBFF7883297DBC}{}
{
  

 
\section{\textcolor{Chapter }{Projective toric varieties: Examples}}\label{Chapter_Projective_toric_varieties_Section_Projective_toric_varieties_Examples}
\logpage{[ 6, 1, 0 ]}
\hyperdef{L}{X8403EE76819E200F}{}
{
  

 
\subsection{\textcolor{Chapter }{P1xP1 created by a polytope}}\label{Chapter_Projective_toric_varieties_Section_Projective_toric_varieties_Examples_Subsection_P1xP1_created_by_a_polytope}
\logpage{[ 6, 1, 1 ]}
\hyperdef{L}{X83FD64C083E2CBC4}{}
{
  
\begin{Verbatim}[commandchars=!@B,fontsize=\small,frame=single,label=Example]
  !gapprompt@gap>B !gapinput@P1P1 := Polytope( [[1,1],[1,-1],[-1,-1],[-1,1]] );B
  <A polytope in |R^2>
  !gapprompt@gap>B !gapinput@P1P1 := ToricVariety( P1P1 );B
  <A projective toric variety of dimension 2>
  !gapprompt@gap>B !gapinput@IsProjective( P1P1 );B
  true
  !gapprompt@gap>B !gapinput@IsComplete( P1P1 );B
  true 
  !gapprompt@gap>B !gapinput@CoordinateRingOfTorus( P1P1, "x" );B
  Q[x1,x1_,x2,x2_]/( x1*x1_-1, x2*x2_-1 )
  !gapprompt@gap>B !gapinput@IsVeryAmple( Polytope( P1P1 ) );B
  true
  !gapprompt@gap>B !gapinput@ProjectiveEmbedding( P1P1 );B
  [ |[ x1_*x2_ ]|, |[ x1_ ]|, |[ x1_*x2 ]|, |[ x2_ ]|,
  |[ 1 ]|, |[ x2 ]|, |[ x1*x2_ ]|, |[ x1 ]|, |[ x1*x2 ]| ]
  !gapprompt@gap>B !gapinput@Length( ProjectiveEmbedding( P1P1 ) );B
  9
  !gapprompt@gap>B !gapinput@CoxRing( P1P1 );B
  Q[x_1,x_2,x_3,x_4]
  (weights: [ ( 0, 1 ), ( 1, 0 ), ( 1, 0 ), ( 0, 1 ) ])
  !gapprompt@gap>B !gapinput@Display( SRIdeal( P1P1 ) );B
  x_1*x_4,
  x_2*x_3 
  
  A (left) ideal generated by the 2 entries of the above matrix
  
  (graded, degrees of generators: [ ( 0, 2 ), ( 2, 0 ) ])
  !gapprompt@gap>B !gapinput@Display( IrrelevantIdeal( P1P1 ) );B
  x_1*x_2,
  x_1*x_3,
  x_2*x_4,
  x_3*x_4 
  
  A (left) ideal generated by the 4 entries of the above matrix
  
  (graded, degrees of generators: [ ( 1, 1 ), ( 1, 1 ), ( 1, 1 ), ( 1, 1 ) ])
\end{Verbatim}
 

 }

 
\subsection{\textcolor{Chapter }{P1xP1 from product of P1s}}\label{Chapter_Projective_toric_varieties_Section_Projective_toric_varieties_Examples_Subsection_P1xP1_from_product_of_P1s}
\logpage{[ 6, 1, 2 ]}
\hyperdef{L}{X7CE9FDAE7AA1C8C3}{}
{
  
\begin{Verbatim}[commandchars=!@|,fontsize=\small,frame=single,label=Example]
  !gapprompt@gap>| !gapinput@P1 := ProjectiveSpace( 1 );|
  <A projective toric variety of dimension 1>
  !gapprompt@gap>| !gapinput@IsComplete( P1 );|
  true
  !gapprompt@gap>| !gapinput@IsSmooth( P1 );|
  true
  !gapprompt@gap>| !gapinput@Dimension( P1 );|
  1
  !gapprompt@gap>| !gapinput@CoxRing( P1, "q" );|
  Q[q_1,q_2]
  (weights: [ 1, 1 ])
  !gapprompt@gap>| !gapinput@P1xP1 := P1*P1;|
  <A projective smooth toric variety of dimension 2 which is a product 
  of 2 toric varieties>
  !gapprompt@gap>| !gapinput@ByASmallerPresentation( ClassGroup( P1xP1 ) );|
  <A free left module of rank 2 on free generators>
  !gapprompt@gap>| !gapinput@CoxRing( P1xP1, "x1,y1,y2,x2" );|
  Q[x1,y1,y2,x2]
  (weights: [ ( 0, 1 ), ( 1, 0 ), ( 1, 0 ), ( 0, 1 ) ])
  !gapprompt@gap>| !gapinput@Display( SRIdeal( P1xP1 ) );|
  x1*x2,
  y1*y2
  
  A (left) ideal generated by the 2 entries of the above matrix
  
  (graded, degrees of generators: [ ( 0, 2 ), ( 2, 0 ) ])
  !gapprompt@gap>| !gapinput@Display( IrrelevantIdeal( P1xP1 ) );|
  x1*y1,
  x1*y2,
  y1*x2,
  y2*x2
  
  A (left) ideal generated by the 4 entries of the above matrix
  
  (graded, degrees of generators: [ ( 1, 1 ), ( 1, 1 ), ( 1, 1 ), ( 1, 1 ) ])
\end{Verbatim}
 

 }

 }

 
\section{\textcolor{Chapter }{The GAP category}}\label{Chapter_Projective_toric_varieties_Section_The_GAP_category}
\logpage{[ 6, 2, 0 ]}
\hyperdef{L}{X7819115082F3E2A6}{}
{
  

\subsection{\textcolor{Chapter }{IsProjectiveToricVariety (for IsToricVariety)}}
\logpage{[ 6, 2, 1 ]}\nobreak
\hyperdef{L}{X7AF9D03A849061DA}{}
{\noindent\textcolor{FuncColor}{$\triangleright$\enspace\texttt{IsProjectiveToricVariety({\mdseries\slshape M})\index{IsProjectiveToricVariety@\texttt{IsProjectiveToricVariety}!for IsToricVariety}
\label{IsProjectiveToricVariety:for IsToricVariety}
}\hfill{\scriptsize (filter)}}\\
\textbf{\indent Returns:\ }
true or false 



 The \mbox{\texttt{\mdseries\slshape GAP}} category of a projective toric variety. }

 }

 
\section{\textcolor{Chapter }{Attribute}}\label{Chapter_Projective_toric_varieties_Section_Attribute}
\logpage{[ 6, 3, 0 ]}
\hyperdef{L}{X7D67342087888E39}{}
{
  

\subsection{\textcolor{Chapter }{PolytopeOfVariety (for IsToricVariety)}}
\logpage{[ 6, 3, 1 ]}\nobreak
\hyperdef{L}{X7FA85FA878557BEA}{}
{\noindent\textcolor{FuncColor}{$\triangleright$\enspace\texttt{PolytopeOfVariety({\mdseries\slshape vari})\index{PolytopeOfVariety@\texttt{PolytopeOfVariety}!for IsToricVariety}
\label{PolytopeOfVariety:for IsToricVariety}
}\hfill{\scriptsize (attribute)}}\\
\textbf{\indent Returns:\ }
a polytope 



 Returns the polytope corresponding to the projective toric variety \mbox{\texttt{\mdseries\slshape vari}}, if it exists. }

 

\subsection{\textcolor{Chapter }{AffineCone (for IsToricVariety)}}
\logpage{[ 6, 3, 2 ]}\nobreak
\hyperdef{L}{X7B42C7097EB11BD2}{}
{\noindent\textcolor{FuncColor}{$\triangleright$\enspace\texttt{AffineCone({\mdseries\slshape vari})\index{AffineCone@\texttt{AffineCone}!for IsToricVariety}
\label{AffineCone:for IsToricVariety}
}\hfill{\scriptsize (attribute)}}\\
\textbf{\indent Returns:\ }
a cone 



 Returns the affine cone of the projective toric variety \mbox{\texttt{\mdseries\slshape vari}}. }

 

\subsection{\textcolor{Chapter }{ProjectiveEmbedding (for IsToricVariety)}}
\logpage{[ 6, 3, 3 ]}\nobreak
\hyperdef{L}{X8582D7EB79B603BA}{}
{\noindent\textcolor{FuncColor}{$\triangleright$\enspace\texttt{ProjectiveEmbedding({\mdseries\slshape vari})\index{ProjectiveEmbedding@\texttt{ProjectiveEmbedding}!for IsToricVariety}
\label{ProjectiveEmbedding:for IsToricVariety}
}\hfill{\scriptsize (attribute)}}\\
\textbf{\indent Returns:\ }
a list 



 Returns characters for a closed embedding in an projective space for the
projective toric variety \mbox{\texttt{\mdseries\slshape vari}}. }

 }

 
\section{\textcolor{Chapter }{Properties}}\label{Chapter_Projective_toric_varieties_Section_Properties}
\logpage{[ 6, 4, 0 ]}
\hyperdef{L}{X871597447BB998A1}{}
{
  

\subsection{\textcolor{Chapter }{IsIsomorphicToProjectiveSpace (for IsToricVariety)}}
\logpage{[ 6, 4, 1 ]}\nobreak
\hyperdef{L}{X8333455D78336442}{}
{\noindent\textcolor{FuncColor}{$\triangleright$\enspace\texttt{IsIsomorphicToProjectiveSpace({\mdseries\slshape vari})\index{IsIsomorphicToProjectiveSpace@\texttt{IsIsomorphicToProjectiveSpace}!for IsToricVariety}
\label{IsIsomorphicToProjectiveSpace:for IsToricVariety}
}\hfill{\scriptsize (property)}}\\
\textbf{\indent Returns:\ }
true or false 



 Checks if the given toric variety \mbox{\texttt{\mdseries\slshape vari}} is a projective space. }

 

\subsection{\textcolor{Chapter }{IsDirectProductOfPNs (for IsToricVariety)}}
\logpage{[ 6, 4, 2 ]}\nobreak
\hyperdef{L}{X7BF350A77D5FE823}{}
{\noindent\textcolor{FuncColor}{$\triangleright$\enspace\texttt{IsDirectProductOfPNs({\mdseries\slshape vari})\index{IsDirectProductOfPNs@\texttt{IsDirectProductOfPNs}!for IsToricVariety}
\label{IsDirectProductOfPNs:for IsToricVariety}
}\hfill{\scriptsize (property)}}\\
\textbf{\indent Returns:\ }
true or false 



 Checks if the given toric variety \mbox{\texttt{\mdseries\slshape vari}} is a direct product of projective spaces. }

 }

 
\section{\textcolor{Chapter }{Methods}}\label{Chapter_Projective_toric_varieties_Section_Methods}
\logpage{[ 6, 5, 0 ]}
\hyperdef{L}{X8606FDCE878850EF}{}
{
  

\subsection{\textcolor{Chapter }{Polytope (for IsToricVariety)}}
\logpage{[ 6, 5, 1 ]}\nobreak
\hyperdef{L}{X7EE3034280F74482}{}
{\noindent\textcolor{FuncColor}{$\triangleright$\enspace\texttt{Polytope({\mdseries\slshape vari})\index{Polytope@\texttt{Polytope}!for IsToricVariety}
\label{Polytope:for IsToricVariety}
}\hfill{\scriptsize (operation)}}\\
\textbf{\indent Returns:\ }
a polytope 



 Returns the polytope of the variety \mbox{\texttt{\mdseries\slshape vari}}. Another name for PolytopeOfVariety for compatibility and shortness. }

 

\subsection{\textcolor{Chapter }{AmpleDivisor (for IsToricVariety and HasPolytopeOfVariety)}}
\logpage{[ 6, 5, 2 ]}\nobreak
\hyperdef{L}{X80E4786583CCB24A}{}
{\noindent\textcolor{FuncColor}{$\triangleright$\enspace\texttt{AmpleDivisor({\mdseries\slshape vari})\index{AmpleDivisor@\texttt{AmpleDivisor}!for IsToricVariety and HasPolytopeOfVariety}
\label{AmpleDivisor:for IsToricVariety and HasPolytopeOfVariety}
}\hfill{\scriptsize (operation)}}\\
\textbf{\indent Returns:\ }
an ample divisor 



 Given a projective toric variety \mbox{\texttt{\mdseries\slshape vari}} constructed from a polytope, this method computes the toric divisor associated
to this polytope. By general theory (see Cox-Schenk-Little) this divisor is
known to be ample. Thus this method computes an ample divisor on the given
toric variety. }

 }

 
\section{\textcolor{Chapter }{Constructors}}\label{Chapter_Projective_toric_varieties_Section_Constructors}
\logpage{[ 6, 6, 0 ]}
\hyperdef{L}{X86EC0F0A78ECBC10}{}
{
  The constructors are the same as for toric varieties. Calling them with a
polytope will result in a projective variety. }

 }

   
\chapter{\textcolor{Chapter }{Toric morphisms}}\label{Chapter_Toric_morphisms}
\logpage{[ 7, 0, 0 ]}
\hyperdef{L}{X7FA18F537F3F5237}{}
{
  

 
\section{\textcolor{Chapter }{Toric morphisms: Examples}}\label{Chapter_Toric_morphisms_Section_Toric_morphisms_Examples}
\logpage{[ 7, 1, 0 ]}
\hyperdef{L}{X83CE579A7B2021DE}{}
{
  

 
\subsection{\textcolor{Chapter }{Morphism between toric varieties and their class groups}}\label{Chapter_Toric_morphisms_Section_Toric_morphisms_Examples_Subsection_Morphism_between_toric_varieties_and_their_class_groups}
\logpage{[ 7, 1, 1 ]}
\hyperdef{L}{X7D1DD8EA8098C432}{}
{
  
\begin{Verbatim}[commandchars=!@E,fontsize=\small,frame=single,label=Example]
  !gapprompt@gap>E !gapinput@P1 := Polytope([[0],[1]]);E
  <A polytope in |R^1>
  !gapprompt@gap>E !gapinput@P2 := Polytope([[0,0],[0,1],[1,0]]);E
  <A polytope in |R^2>
  !gapprompt@gap>E !gapinput@P1 := ToricVariety( P1 );E
  <A projective toric variety of dimension 1>
  !gapprompt@gap>E !gapinput@P2 := ToricVariety( P2 );E
  <A projective toric variety of dimension 2>
  !gapprompt@gap>E !gapinput@P1P2 := P1*P2;E
  <A projective toric variety of dimension 3
   which is a product of 2 toric varieties>
  !gapprompt@gap>E !gapinput@ClassGroup( P1 );E
  <A free left module of rank 1 on a free generator>
  !gapprompt@gap>E !gapinput@Display(ByASmallerPresentation(ClassGroup( P1 )));E
  Z^(1 x 1)
  !gapprompt@gap>E !gapinput@ClassGroup( P2 );E
  <A free left module of rank 1 on a free generator>
  !gapprompt@gap>E !gapinput@Display(ByASmallerPresentation(ClassGroup( P2 )));E
  Z^(1 x 1)
  !gapprompt@gap>E !gapinput@ClassGroup( P1P2 );E
  <A free left module of rank 2 on free generators>
  !gapprompt@gap>E !gapinput@Display( last );E
  Z^(1 x 2)
  !gapprompt@gap>E !gapinput@PicardGroup( P1P2 );E
  <A free left module of rank 2 on free generators>
  !gapprompt@gap>E !gapinput@P1P2;E
  <A projective smooth toric variety of dimension 3 
   which is a product of 2 toric varieties>
  !gapprompt@gap>E !gapinput@P2P1:=P2*P1;E
  <A projective toric variety of dimension 3 
   which is a product of 2 toric varieties>
  !gapprompt@gap>E !gapinput@M := [[0,0,1],[1,0,0],[0,1,0]];E
  [ [ 0, 0, 1 ], [ 1, 0, 0 ], [ 0, 1, 0 ] ]
  !gapprompt@gap>E !gapinput@M := ToricMorphism(P1P2,M,P2P1);E
  <A "homomorphism" of right objects>
  !gapprompt@gap>E !gapinput@IsMorphism(M);E
  true
  !gapprompt@gap>E !gapinput@ClassGroup(M);E
  <A homomorphism of left modules>
  !gapprompt@gap>E !gapinput@Display(ClassGroup(M));E
  [ [  0,  1 ],
    [  1,  0 ] ]
  
  the map is currently represented by the above 2 x 2 matrix
  !gapprompt@gap>E !gapinput@ToricImageObject( M );E
  <A toric variety of dimension 3>
  !gapprompt@gap>E !gapinput@UnderlyingGridMorphism( M );E
  <A homomorphism of left modules>
  !gapprompt@gap>E !gapinput@MorphismOnCartierDivisorGroup( M );E
  <A homomorphism of left modules>
  !gapprompt@gap>E !gapinput@M2 := ToricMorphism( P1P2, [[0,0,1],[1,0,0],[0,1,0]] );E
  <A "homomorphism" of right objects>
  !gapprompt@gap>E !gapinput@IsMorphism( M2 );E
  true
  !gapprompt@gap>E !gapinput@M = M2;E
  false
\end{Verbatim}
 

 }

 }

 
\section{\textcolor{Chapter }{The GAP category}}\label{Chapter_Toric_morphisms_Section_The_GAP_category}
\logpage{[ 7, 2, 0 ]}
\hyperdef{L}{X7819115082F3E2A6}{}
{
  

\subsection{\textcolor{Chapter }{IsToricMorphism (for IsObject)}}
\logpage{[ 7, 2, 1 ]}\nobreak
\hyperdef{L}{X819A80C87F2E520D}{}
{\noindent\textcolor{FuncColor}{$\triangleright$\enspace\texttt{IsToricMorphism({\mdseries\slshape M})\index{IsToricMorphism@\texttt{IsToricMorphism}!for IsObject}
\label{IsToricMorphism:for IsObject}
}\hfill{\scriptsize (filter)}}\\
\textbf{\indent Returns:\ }
true or false 



 The \mbox{\texttt{\mdseries\slshape GAP}} category of toric morphisms. A toric morphism is defined by a grid
homomorphism, which is compatible with the fan structure of the two varieties. }

 }

 
\section{\textcolor{Chapter }{Properties}}\label{Chapter_Toric_morphisms_Section_Properties}
\logpage{[ 7, 3, 0 ]}
\hyperdef{L}{X871597447BB998A1}{}
{
  

\subsection{\textcolor{Chapter }{IsMorphism (for IsToricMorphism)}}
\logpage{[ 7, 3, 1 ]}\nobreak
\hyperdef{L}{X8409007978D5DCE1}{}
{\noindent\textcolor{FuncColor}{$\triangleright$\enspace\texttt{IsMorphism({\mdseries\slshape morph})\index{IsMorphism@\texttt{IsMorphism}!for IsToricMorphism}
\label{IsMorphism:for IsToricMorphism}
}\hfill{\scriptsize (property)}}\\
\textbf{\indent Returns:\ }
true or false 



 Checks if the grid morphism \mbox{\texttt{\mdseries\slshape morph}} respects the fan structure. }

 

\subsection{\textcolor{Chapter }{IsProper (for IsToricMorphism)}}
\logpage{[ 7, 3, 2 ]}\nobreak
\hyperdef{L}{X785098297B93ED95}{}
{\noindent\textcolor{FuncColor}{$\triangleright$\enspace\texttt{IsProper({\mdseries\slshape morph})\index{IsProper@\texttt{IsProper}!for IsToricMorphism}
\label{IsProper:for IsToricMorphism}
}\hfill{\scriptsize (property)}}\\
\textbf{\indent Returns:\ }
true or false 



 Checks if the defined morphism \mbox{\texttt{\mdseries\slshape morph}} is proper. }

 }

 
\section{\textcolor{Chapter }{Attributes}}\label{Chapter_Toric_morphisms_Section_Attributes}
\logpage{[ 7, 4, 0 ]}
\hyperdef{L}{X7C701DBF7BAE649A}{}
{
  

\subsection{\textcolor{Chapter }{SourceObject (for IsToricMorphism)}}
\logpage{[ 7, 4, 1 ]}\nobreak
\hyperdef{L}{X814F0ED07CB1C305}{}
{\noindent\textcolor{FuncColor}{$\triangleright$\enspace\texttt{SourceObject({\mdseries\slshape morph})\index{SourceObject@\texttt{SourceObject}!for IsToricMorphism}
\label{SourceObject:for IsToricMorphism}
}\hfill{\scriptsize (attribute)}}\\
\textbf{\indent Returns:\ }
a variety 



 Returns the source object of the morphism \mbox{\texttt{\mdseries\slshape morph}}. This attribute is a must have. }

 

\subsection{\textcolor{Chapter }{UnderlyingGridMorphism (for IsToricMorphism)}}
\logpage{[ 7, 4, 2 ]}\nobreak
\hyperdef{L}{X831FCE4180C361F7}{}
{\noindent\textcolor{FuncColor}{$\triangleright$\enspace\texttt{UnderlyingGridMorphism({\mdseries\slshape morph})\index{UnderlyingGridMorphism@\texttt{UnderlyingGridMorphism}!for IsToricMorphism}
\label{UnderlyingGridMorphism:for IsToricMorphism}
}\hfill{\scriptsize (attribute)}}\\
\textbf{\indent Returns:\ }
a map 



 Returns the grid map which defines \mbox{\texttt{\mdseries\slshape morph}}. }

 

\subsection{\textcolor{Chapter }{ToricImageObject (for IsToricMorphism)}}
\logpage{[ 7, 4, 3 ]}\nobreak
\hyperdef{L}{X79348EE07FE1404B}{}
{\noindent\textcolor{FuncColor}{$\triangleright$\enspace\texttt{ToricImageObject({\mdseries\slshape morph})\index{ToricImageObject@\texttt{ToricImageObject}!for IsToricMorphism}
\label{ToricImageObject:for IsToricMorphism}
}\hfill{\scriptsize (attribute)}}\\
\textbf{\indent Returns:\ }
a variety 



 Returns the variety which is created by the fan which is the image of the fan
of the source of \mbox{\texttt{\mdseries\slshape morph}}. This is not an image in the usual sense, but a toric image. }

 

\subsection{\textcolor{Chapter }{RangeObject (for IsToricMorphism)}}
\logpage{[ 7, 4, 4 ]}\nobreak
\hyperdef{L}{X7E28EB8A7AA75A90}{}
{\noindent\textcolor{FuncColor}{$\triangleright$\enspace\texttt{RangeObject({\mdseries\slshape morph})\index{RangeObject@\texttt{RangeObject}!for IsToricMorphism}
\label{RangeObject:for IsToricMorphism}
}\hfill{\scriptsize (attribute)}}\\
\textbf{\indent Returns:\ }
a variety 



 Returns the range of the morphism \mbox{\texttt{\mdseries\slshape morph}}. If no range is given (yes, this is possible), the method returns the image. }

 

\subsection{\textcolor{Chapter }{MorphismOnWeilDivisorGroup (for IsToricMorphism)}}
\logpage{[ 7, 4, 5 ]}\nobreak
\hyperdef{L}{X7A79548C7DD721E6}{}
{\noindent\textcolor{FuncColor}{$\triangleright$\enspace\texttt{MorphismOnWeilDivisorGroup({\mdseries\slshape morph})\index{MorphismOnWeilDivisorGroup@\texttt{MorphismOnWeilDivisorGroup}!for IsToricMorphism}
\label{MorphismOnWeilDivisorGroup:for IsToricMorphism}
}\hfill{\scriptsize (attribute)}}\\
\textbf{\indent Returns:\ }
a morphism 



 Returns the associated morphism between the divisor group of the range of \mbox{\texttt{\mdseries\slshape morph}} and the divisor group of the source. }

 

\subsection{\textcolor{Chapter }{ClassGroup (for IsToricMorphism)}}
\logpage{[ 7, 4, 6 ]}\nobreak
\hyperdef{L}{X85593D208343F9A9}{}
{\noindent\textcolor{FuncColor}{$\triangleright$\enspace\texttt{ClassGroup({\mdseries\slshape morph})\index{ClassGroup@\texttt{ClassGroup}!for IsToricMorphism}
\label{ClassGroup:for IsToricMorphism}
}\hfill{\scriptsize (attribute)}}\\
\textbf{\indent Returns:\ }
a morphism 



 Returns the associated morphism between the class groups of source and range
of the morphism \mbox{\texttt{\mdseries\slshape morph}} }

 

\subsection{\textcolor{Chapter }{MorphismOnCartierDivisorGroup (for IsToricMorphism)}}
\logpage{[ 7, 4, 7 ]}\nobreak
\hyperdef{L}{X86DD03A485FBE855}{}
{\noindent\textcolor{FuncColor}{$\triangleright$\enspace\texttt{MorphismOnCartierDivisorGroup({\mdseries\slshape morph})\index{MorphismOnCartierDivisorGroup@\texttt{MorphismOnCartierDivisorGroup}!for IsToricMorphism}
\label{MorphismOnCartierDivisorGroup:for IsToricMorphism}
}\hfill{\scriptsize (attribute)}}\\
\textbf{\indent Returns:\ }
a morphism 



 Returns the associated morphism between the Cartier divisor groups of source
and range of the morphism \mbox{\texttt{\mdseries\slshape morph}} }

 

\subsection{\textcolor{Chapter }{PicardGroup (for IsToricMorphism)}}
\logpage{[ 7, 4, 8 ]}\nobreak
\hyperdef{L}{X7A4613747EB37387}{}
{\noindent\textcolor{FuncColor}{$\triangleright$\enspace\texttt{PicardGroup({\mdseries\slshape morph})\index{PicardGroup@\texttt{PicardGroup}!for IsToricMorphism}
\label{PicardGroup:for IsToricMorphism}
}\hfill{\scriptsize (attribute)}}\\
\textbf{\indent Returns:\ }
a morphism 



 Returns the associated morphism between the Picard groups of source and range
of the morphism \mbox{\texttt{\mdseries\slshape morph}} }

 

\subsection{\textcolor{Chapter }{Source (for IsToricMorphism)}}
\logpage{[ 7, 4, 9 ]}\nobreak
\hyperdef{L}{X7C22BDE88295E1A4}{}
{\noindent\textcolor{FuncColor}{$\triangleright$\enspace\texttt{Source({\mdseries\slshape morph})\index{Source@\texttt{Source}!for IsToricMorphism}
\label{Source:for IsToricMorphism}
}\hfill{\scriptsize (attribute)}}\\
\textbf{\indent Returns:\ }
a variety 



 Return the source of the toric morphism \mbox{\texttt{\mdseries\slshape morph}}. }

 

\subsection{\textcolor{Chapter }{Range (for IsToricMorphism)}}
\logpage{[ 7, 4, 10 ]}\nobreak
\hyperdef{L}{X79A63E2585025A8C}{}
{\noindent\textcolor{FuncColor}{$\triangleright$\enspace\texttt{Range({\mdseries\slshape morph})\index{Range@\texttt{Range}!for IsToricMorphism}
\label{Range:for IsToricMorphism}
}\hfill{\scriptsize (attribute)}}\\
\textbf{\indent Returns:\ }
a variety 



 Returns the range of the toric morphism \mbox{\texttt{\mdseries\slshape morph}} if specified. }

 

\subsection{\textcolor{Chapter }{MorphismOnIthFactor (for IsToricMorphism)}}
\logpage{[ 7, 4, 11 ]}\nobreak
\hyperdef{L}{X7E855CA87D388FCC}{}
{\noindent\textcolor{FuncColor}{$\triangleright$\enspace\texttt{MorphismOnIthFactor({\mdseries\slshape morph})\index{MorphismOnIthFactor@\texttt{MorphismOnIthFactor}!for IsToricMorphism}
\label{MorphismOnIthFactor:for IsToricMorphism}
}\hfill{\scriptsize (attribute)}}\\


 

 }

 }

 
\section{\textcolor{Chapter }{Methods}}\label{Chapter_Toric_morphisms_Section_Methods}
\logpage{[ 7, 5, 0 ]}
\hyperdef{L}{X8606FDCE878850EF}{}
{
  

\subsection{\textcolor{Chapter }{UnderlyingListList (for IsToricMorphism)}}
\logpage{[ 7, 5, 1 ]}\nobreak
\hyperdef{L}{X8159CA087F1A607B}{}
{\noindent\textcolor{FuncColor}{$\triangleright$\enspace\texttt{UnderlyingListList({\mdseries\slshape morph})\index{UnderlyingListList@\texttt{UnderlyingListList}!for IsToricMorphism}
\label{UnderlyingListList:for IsToricMorphism}
}\hfill{\scriptsize (operation)}}\\
\textbf{\indent Returns:\ }
a list 



 Returns a list of list which represents the grid homomorphism. }

 }

 
\section{\textcolor{Chapter }{Constructors}}\label{Chapter_Toric_morphisms_Section_Constructors}
\logpage{[ 7, 6, 0 ]}
\hyperdef{L}{X86EC0F0A78ECBC10}{}
{
  

\subsection{\textcolor{Chapter }{ToricMorphism (for IsToricVariety, IsList)}}
\logpage{[ 7, 6, 1 ]}\nobreak
\hyperdef{L}{X7C6ADA8E7911025F}{}
{\noindent\textcolor{FuncColor}{$\triangleright$\enspace\texttt{ToricMorphism({\mdseries\slshape vari, lis})\index{ToricMorphism@\texttt{ToricMorphism}!for IsToricVariety, IsList}
\label{ToricMorphism:for IsToricVariety, IsList}
}\hfill{\scriptsize (operation)}}\\
\textbf{\indent Returns:\ }
a morphism 



 Returns the toric morphism with source \mbox{\texttt{\mdseries\slshape vari}} which is represented by the matrix \mbox{\texttt{\mdseries\slshape lis}}. The range is set to the image. }

 

\subsection{\textcolor{Chapter }{ToricMorphism (for IsToricVariety, IsList, IsToricVariety)}}
\logpage{[ 7, 6, 2 ]}\nobreak
\hyperdef{L}{X82AE442981070044}{}
{\noindent\textcolor{FuncColor}{$\triangleright$\enspace\texttt{ToricMorphism({\mdseries\slshape vari, lis, vari2})\index{ToricMorphism@\texttt{ToricMorphism}!for IsToricVariety, IsList, IsToricVariety}
\label{ToricMorphism:for IsToricVariety, IsList, IsToricVariety}
}\hfill{\scriptsize (operation)}}\\
\textbf{\indent Returns:\ }
a morphism 



 Returns the toric morphism with source \mbox{\texttt{\mdseries\slshape vari}} and range \mbox{\texttt{\mdseries\slshape vari2}} which is represented by the matrix \mbox{\texttt{\mdseries\slshape lis}}. }

 }

 }

   
\chapter{\textcolor{Chapter }{Toric divisors}}\label{Chapter_Toric_divisors}
\logpage{[ 8, 0, 0 ]}
\hyperdef{L}{X7FDB76897833225A}{}
{
  

 
\section{\textcolor{Chapter }{Toric divisors: Examples}}\label{Chapter_Toric_divisors_Section_Toric_divisors_Examples}
\logpage{[ 8, 1, 0 ]}
\hyperdef{L}{X7FE84D6F87076DA0}{}
{
  

 
\subsection{\textcolor{Chapter }{Divisors on a toric variety}}\label{Chapter_Toric_divisors_Section_Toric_divisors_Examples_Subsection_Divisors_on_a_toric_variety}
\logpage{[ 8, 1, 1 ]}
\hyperdef{L}{X7A7080E77E93A36F}{}
{
  
\begin{Verbatim}[commandchars=!@J,fontsize=\small,frame=single,label=Example]
  !gapprompt@gap>J !gapinput@H7 := Fan( [[0,1],[1,0],[0,-1],[-1,7]],[[1,2],[2,3],[3,4],[4,1]] );J
  <A fan in |R^2>
  !gapprompt@gap>J !gapinput@H7 := ToricVariety( H7 );J
  <A toric variety of dimension 2>
  !gapprompt@gap>J !gapinput@P := TorusInvariantPrimeDivisors( H7 );J
  [ <A prime divisor of a toric variety with coordinates ( 1, 0, 0, 0 )>,
    <A prime divisor of a toric variety with coordinates ( 0, 1, 0, 0 )>,
    <A prime divisor of a toric variety with coordinates ( 0, 0, 1, 0 )>,
    <A prime divisor of a toric variety with coordinates ( 0, 0, 0, 1 )> ]
  !gapprompt@gap>J !gapinput@D := P[1]+P[2];J
  <A divisor of a toric variety with coordinates ( 1, 1, 0, 0 )>
  !gapprompt@gap>J !gapinput@IsBasepointFree(D);J
  true
  !gapprompt@gap>J !gapinput@IsAmple(D);J
  true
  !gapprompt@gap>J !gapinput@CoordinateRingOfTorus(H7,"x");J
  Q[x1,x1_,x2,x2_]/( x1*x1_-1, x2*x2_-1 )
  !gapprompt@gap>J !gapinput@Polytope(D);J
  <A polytope in |R^2>
  !gapprompt@gap>J !gapinput@CharactersForClosedEmbedding(D);J
  [ |[ 1 ]|, |[ x2 ]|, |[ x1 ]|, |[ x1*x2 ]|, |[ x1^2*x2 ]|, 
    |[ x1^3*x2 ]|, |[ x1^4*x2 ]|, |[ x1^5*x2 ]|, 
    |[ x1^6*x2 ]|, |[ x1^7*x2 ]|, |[ x1^8*x2 ]| ]
  !gapprompt@gap>J !gapinput@CoxRingOfTargetOfDivisorMorphism(D);J
  Q[x_1,x_2,x_3,x_4,x_5,x_6,x_7,x_8,x_9,x_10,x_11]
  (weights: [ 1, 1, 1, 1, 1, 1, 1, 1, 1, 1, 1 ])
  !gapprompt@gap>J !gapinput@RingMorphismOfDivisor(D);J
  <A "homomorphism" of rings>
  !gapprompt@gap>J !gapinput@Display(RingMorphismOfDivisor(D));J
  Q[x_1,x_2,x_3,x_4]
  (weights: [ ( 0, 1 ), ( 1, 0 ), ( 1, -7 ), ( 0, 1 ) ])
    ^
    |
  [ x_1*x_2, x_1^8*x_3, x_2*x_4, x_1^7*x_3*x_4, x_1^6*x_3*x_4^2, 
    x_1^5*x_3*x_4^3, x_1^4*x_3*x_4^4, x_1^3*x_3*x_4^5, x_1^2*x_3*x_4^6, 
    x_1*x_3*x_4^7, x_3*x_4^8 ]
    |
    |
  Q[x_1,x_2,x_3,x_4,x_5,x_6,x_7,x_8,x_9,x_10,x_11]
  (weights: [ 1, 1, 1, 1, 1, 1, 1, 1, 1, 1, 1 ])
  !gapprompt@gap>J !gapinput@ByASmallerPresentation(ClassGroup(H7));J
  <A free left module of rank 2 on free generators>
  !gapprompt@gap>J !gapinput@MonomsOfCoxRingOfDegree(D);J
  [ x_1*x_2, x_1^8*x_3, x_2*x_4, x_1^7*x_3*x_4, x_1^6*x_3*x_4^2, 
    x_1^5*x_3*x_4^3, x_1^4*x_3*x_4^4, x_1^3*x_3*x_4^5, x_1^2*x_3*x_4^6, 
    x_1*x_3*x_4^7, x_3*x_4^8 ]
  !gapprompt@gap>J !gapinput@D2:=D-2*P[2];J
  <A divisor of a toric variety with coordinates ( 1, -1, 0, 0 )>
  !gapprompt@gap>J !gapinput@D = D2;J
  false
  !gapprompt@gap>J !gapinput@IsBasepointFree(D2);J
  false
  !gapprompt@gap>J !gapinput@IsAmple(D2);J
  false
\end{Verbatim}
 
\begin{Verbatim}[commandchars=!@|,fontsize=\small,frame=single,label=Example]
  !gapprompt@gap>| !gapinput@P2 := ProjectiveSpace( 2 );|
  <A projective toric variety of dimension 2>
  !gapprompt@gap>| !gapinput@CoxRing( P2 );|
  Q[x_1,x_2,x_3]
  (weights: [ 1, 1, 1 ])
  !gapprompt@gap>| !gapinput@DA := AmpleDivisor( P2 );|
  <A divisor of a toric variety with coordinates ( 1, 0, 0 )>
  !gapprompt@gap>| !gapinput@IsPrincipal( DA );|
  false
  !gapprompt@gap>| !gapinput@IsPrimedivisor( DA );|
  true
  !gapprompt@gap>| !gapinput@IsAmple( DA );|
  true
  !gapprompt@gap>| !gapinput@IsToricDivisor( DA );|
  true
  !gapprompt@gap>| !gapinput@IsBasepointFree( DA );|
  true
  !gapprompt@gap>| !gapinput@IntegerForWhichIsSureVeryAmple( DA );|
  1
  !gapprompt@gap>| !gapinput@UnderlyingToricVariety( DA );|
  <A toric subvariety of dimension 1>
  !gapprompt@gap>| !gapinput@DegreeOfDivisor( DA );|
  1
  !gapprompt@gap>| !gapinput@Display( DA );|
  An ample basepoint free Cartier divisor of a toric variety.
  !gapprompt@gap>| !gapinput@ViewObj( DA );|
  <An ample basepoint free Cartier prime divisor of a toric variety with coordinates ( 1, 0, 0 )>
\end{Verbatim}
 

 }

 
\subsection{\textcolor{Chapter }{Polytope of toric divisors}}\label{Chapter_Toric_divisors_Section_Toric_divisors_Examples_Subsection_Polytope_of_toric_divisors}
\logpage{[ 8, 1, 2 ]}
\hyperdef{L}{X7AA4909E8359A584}{}
{
  
\begin{Verbatim}[commandchars=!@B,fontsize=\small,frame=single,label=Example]
  !gapprompt@gap>B !gapinput@P1 := ProjectiveSpace( 1 );B
  <A projective toric variety of dimension 1>
  !gapprompt@gap>B !gapinput@divisor := DivisorOfGivenClass( P1, [ -1 ] );B
  <A divisor of a toric variety with coordinates ( -1, 0 )>
  !gapprompt@gap>B !gapinput@polytope := PolytopeOfDivisor( divisor );B
  <A polytope in |R^1>
\end{Verbatim}
 }

 }

 
\section{\textcolor{Chapter }{The GAP category}}\label{Chapter_Toric_divisors_Section_The_GAP_category}
\logpage{[ 8, 2, 0 ]}
\hyperdef{L}{X7819115082F3E2A6}{}
{
  

\subsection{\textcolor{Chapter }{IsToricDivisor (for IsObject)}}
\logpage{[ 8, 2, 1 ]}\nobreak
\hyperdef{L}{X86DC31368264C30A}{}
{\noindent\textcolor{FuncColor}{$\triangleright$\enspace\texttt{IsToricDivisor({\mdseries\slshape M})\index{IsToricDivisor@\texttt{IsToricDivisor}!for IsObject}
\label{IsToricDivisor:for IsObject}
}\hfill{\scriptsize (filter)}}\\
\textbf{\indent Returns:\ }
true or false 



 The \mbox{\texttt{\mdseries\slshape GAP}} category of torus invariant Weil divisors. }

 

 

\subsection{\textcolor{Chapter }{twitter (for IsToricDivisor)}}
\logpage{[ 8, 2, 2 ]}\nobreak
\hyperdef{L}{X7E2187387D71480B}{}
{\noindent\textcolor{FuncColor}{$\triangleright$\enspace\texttt{twitter({\mdseries\slshape arg})\index{twitter@\texttt{twitter}!for IsToricDivisor}
\label{twitter:for IsToricDivisor}
}\hfill{\scriptsize (attribute)}}\\


 

 }

 }

 
\section{\textcolor{Chapter }{Properties}}\label{Chapter_Toric_divisors_Section_Properties}
\logpage{[ 8, 3, 0 ]}
\hyperdef{L}{X871597447BB998A1}{}
{
  

\subsection{\textcolor{Chapter }{IsCartier (for IsToricDivisor)}}
\logpage{[ 8, 3, 1 ]}\nobreak
\hyperdef{L}{X85B6624A78CFA656}{}
{\noindent\textcolor{FuncColor}{$\triangleright$\enspace\texttt{IsCartier({\mdseries\slshape divi})\index{IsCartier@\texttt{IsCartier}!for IsToricDivisor}
\label{IsCartier:for IsToricDivisor}
}\hfill{\scriptsize (property)}}\\
\textbf{\indent Returns:\ }
true or false 



 Checks if the torus invariant Weil divisor \mbox{\texttt{\mdseries\slshape divi}} is Cartier i.e. if it is locally principal. }

 

\subsection{\textcolor{Chapter }{IsPrincipal (for IsToricDivisor)}}
\logpage{[ 8, 3, 2 ]}\nobreak
\hyperdef{L}{X7C4E41557FDB17B3}{}
{\noindent\textcolor{FuncColor}{$\triangleright$\enspace\texttt{IsPrincipal({\mdseries\slshape divi})\index{IsPrincipal@\texttt{IsPrincipal}!for IsToricDivisor}
\label{IsPrincipal:for IsToricDivisor}
}\hfill{\scriptsize (property)}}\\
\textbf{\indent Returns:\ }
true or false 



 Checks if the torus invariant Weil divisor \mbox{\texttt{\mdseries\slshape divi}} is principal which in the toric invariant case means that it is the divisor of
a character. }

 

\subsection{\textcolor{Chapter }{IsPrimedivisor (for IsToricDivisor)}}
\logpage{[ 8, 3, 3 ]}\nobreak
\hyperdef{L}{X87877F5D79A0D48C}{}
{\noindent\textcolor{FuncColor}{$\triangleright$\enspace\texttt{IsPrimedivisor({\mdseries\slshape divi})\index{IsPrimedivisor@\texttt{IsPrimedivisor}!for IsToricDivisor}
\label{IsPrimedivisor:for IsToricDivisor}
}\hfill{\scriptsize (property)}}\\
\textbf{\indent Returns:\ }
true or false 



 Checks if the Weil divisor \mbox{\texttt{\mdseries\slshape divi}} represents a prime divisor, i.e. if it is a standard generator of the divisor
group. }

 

\subsection{\textcolor{Chapter }{IsBasepointFree (for IsToricDivisor)}}
\logpage{[ 8, 3, 4 ]}\nobreak
\hyperdef{L}{X7A4D10EE7A939642}{}
{\noindent\textcolor{FuncColor}{$\triangleright$\enspace\texttt{IsBasepointFree({\mdseries\slshape divi})\index{IsBasepointFree@\texttt{IsBasepointFree}!for IsToricDivisor}
\label{IsBasepointFree:for IsToricDivisor}
}\hfill{\scriptsize (property)}}\\
\textbf{\indent Returns:\ }
true or false 



 Checks if the divisor \mbox{\texttt{\mdseries\slshape divi}} is basepoint free. }

 

\subsection{\textcolor{Chapter }{IsAmple (for IsToricDivisor)}}
\logpage{[ 8, 3, 5 ]}\nobreak
\hyperdef{L}{X784E978A7838A358}{}
{\noindent\textcolor{FuncColor}{$\triangleright$\enspace\texttt{IsAmple({\mdseries\slshape divi})\index{IsAmple@\texttt{IsAmple}!for IsToricDivisor}
\label{IsAmple:for IsToricDivisor}
}\hfill{\scriptsize (property)}}\\
\textbf{\indent Returns:\ }
true or false 



 Checks if the divisor \mbox{\texttt{\mdseries\slshape divi}} is ample, i.e. if it is colored red, yellow and green. }

 

\subsection{\textcolor{Chapter }{IsVeryAmple (for IsToricDivisor)}}
\logpage{[ 8, 3, 6 ]}\nobreak
\hyperdef{L}{X85F98B807A07756F}{}
{\noindent\textcolor{FuncColor}{$\triangleright$\enspace\texttt{IsVeryAmple({\mdseries\slshape divi})\index{IsVeryAmple@\texttt{IsVeryAmple}!for IsToricDivisor}
\label{IsVeryAmple:for IsToricDivisor}
}\hfill{\scriptsize (property)}}\\
\textbf{\indent Returns:\ }
true or false 



 Checks if the divisor \mbox{\texttt{\mdseries\slshape divi}} is very ample. }

 

\subsection{\textcolor{Chapter }{IsNumericallyEffective (for IsToricDivisor)}}
\logpage{[ 8, 3, 7 ]}\nobreak
\hyperdef{L}{X83EBA3F0852F61FC}{}
{\noindent\textcolor{FuncColor}{$\triangleright$\enspace\texttt{IsNumericallyEffective({\mdseries\slshape divi})\index{IsNumericallyEffective@\texttt{IsNumericallyEffective}!for IsToricDivisor}
\label{IsNumericallyEffective:for IsToricDivisor}
}\hfill{\scriptsize (property)}}\\
\textbf{\indent Returns:\ }
true or false 



 Checks if the divisor \mbox{\texttt{\mdseries\slshape divi}} is nef. }

 }

 
\section{\textcolor{Chapter }{Attributes}}\label{Chapter_Toric_divisors_Section_Attributes}
\logpage{[ 8, 4, 0 ]}
\hyperdef{L}{X7C701DBF7BAE649A}{}
{
  

\subsection{\textcolor{Chapter }{CartierData (for IsToricDivisor)}}
\logpage{[ 8, 4, 1 ]}\nobreak
\hyperdef{L}{X7C9D1DF87C2BC4AB}{}
{\noindent\textcolor{FuncColor}{$\triangleright$\enspace\texttt{CartierData({\mdseries\slshape divi})\index{CartierData@\texttt{CartierData}!for IsToricDivisor}
\label{CartierData:for IsToricDivisor}
}\hfill{\scriptsize (attribute)}}\\
\textbf{\indent Returns:\ }
a list 



 Returns the Cartier data of the divisor \mbox{\texttt{\mdseries\slshape divi}}, if it is Cartier, and fails otherwise. }

 

\subsection{\textcolor{Chapter }{CharacterOfPrincipalDivisor (for IsToricDivisor)}}
\logpage{[ 8, 4, 2 ]}\nobreak
\hyperdef{L}{X82EBFD32813166B7}{}
{\noindent\textcolor{FuncColor}{$\triangleright$\enspace\texttt{CharacterOfPrincipalDivisor({\mdseries\slshape divi})\index{CharacterOfPrincipalDivisor@\texttt{CharacterOfPrincipalDivisor}!for IsToricDivisor}
\label{CharacterOfPrincipalDivisor:for IsToricDivisor}
}\hfill{\scriptsize (attribute)}}\\
\textbf{\indent Returns:\ }
a homalg module element 



 Returns the character corresponding to the principal divisor \mbox{\texttt{\mdseries\slshape divi}}. }

 

\subsection{\textcolor{Chapter }{ClassOfDivisor (for IsToricDivisor)}}
\logpage{[ 8, 4, 3 ]}\nobreak
\hyperdef{L}{X7BDB2B61844AA35D}{}
{\noindent\textcolor{FuncColor}{$\triangleright$\enspace\texttt{ClassOfDivisor({\mdseries\slshape divi})\index{ClassOfDivisor@\texttt{ClassOfDivisor}!for IsToricDivisor}
\label{ClassOfDivisor:for IsToricDivisor}
}\hfill{\scriptsize (attribute)}}\\
\textbf{\indent Returns:\ }
a homalg module element 



 Returns the class group element corresponding to the divisor \mbox{\texttt{\mdseries\slshape divi}}. }

 

\subsection{\textcolor{Chapter }{PolytopeOfDivisor (for IsToricDivisor)}}
\logpage{[ 8, 4, 4 ]}\nobreak
\hyperdef{L}{X8333191283099DD6}{}
{\noindent\textcolor{FuncColor}{$\triangleright$\enspace\texttt{PolytopeOfDivisor({\mdseries\slshape divi})\index{PolytopeOfDivisor@\texttt{PolytopeOfDivisor}!for IsToricDivisor}
\label{PolytopeOfDivisor:for IsToricDivisor}
}\hfill{\scriptsize (attribute)}}\\
\textbf{\indent Returns:\ }
a polytope 



 Returns the polytope corresponding to the divisor \mbox{\texttt{\mdseries\slshape divi}}. }

 

\subsection{\textcolor{Chapter }{BasisOfGlobalSections (for IsToricDivisor)}}
\logpage{[ 8, 4, 5 ]}\nobreak
\hyperdef{L}{X803FA0C780871883}{}
{\noindent\textcolor{FuncColor}{$\triangleright$\enspace\texttt{BasisOfGlobalSections({\mdseries\slshape divi})\index{BasisOfGlobalSections@\texttt{BasisOfGlobalSections}!for IsToricDivisor}
\label{BasisOfGlobalSections:for IsToricDivisor}
}\hfill{\scriptsize (attribute)}}\\
\textbf{\indent Returns:\ }
a list 



 Returns a basis of the global section module of the quasi-coherent sheaf of
the divisor \mbox{\texttt{\mdseries\slshape divi}}. }

 

\subsection{\textcolor{Chapter }{IntegerForWhichIsSureVeryAmple (for IsToricDivisor)}}
\logpage{[ 8, 4, 6 ]}\nobreak
\hyperdef{L}{X814933247B5D6242}{}
{\noindent\textcolor{FuncColor}{$\triangleright$\enspace\texttt{IntegerForWhichIsSureVeryAmple({\mdseries\slshape divi})\index{IntegerForWhichIsSureVeryAmple@\texttt{IntegerForWhichIsSureVeryAmple}!for IsToricDivisor}
\label{IntegerForWhichIsSureVeryAmple:for IsToricDivisor}
}\hfill{\scriptsize (attribute)}}\\
\textbf{\indent Returns:\ }
an integer 



 Returns an integer \mbox{\texttt{\mdseries\slshape n}} such that $n \cdot $divi is very ample. }

 

\subsection{\textcolor{Chapter }{AmbientToricVariety (for IsToricDivisor)}}
\logpage{[ 8, 4, 7 ]}\nobreak
\hyperdef{L}{X7929762F79ABFA0D}{}
{\noindent\textcolor{FuncColor}{$\triangleright$\enspace\texttt{AmbientToricVariety({\mdseries\slshape divi})\index{AmbientToricVariety@\texttt{AmbientToricVariety}!for IsToricDivisor}
\label{AmbientToricVariety:for IsToricDivisor}
}\hfill{\scriptsize (attribute)}}\\
\textbf{\indent Returns:\ }
a variety 



 Returns the toric variety which contains the prime divisors of the divisor \mbox{\texttt{\mdseries\slshape divi}}. }

 

\subsection{\textcolor{Chapter }{UnderlyingGroupElement (for IsToricDivisor)}}
\logpage{[ 8, 4, 8 ]}\nobreak
\hyperdef{L}{X86DCAD21785C0FA5}{}
{\noindent\textcolor{FuncColor}{$\triangleright$\enspace\texttt{UnderlyingGroupElement({\mdseries\slshape divi})\index{UnderlyingGroupElement@\texttt{UnderlyingGroupElement}!for IsToricDivisor}
\label{UnderlyingGroupElement:for IsToricDivisor}
}\hfill{\scriptsize (attribute)}}\\
\textbf{\indent Returns:\ }
a homalg module element 



 Returns an element which represents the divisor \mbox{\texttt{\mdseries\slshape divi}} in the Weil group. }

 

\subsection{\textcolor{Chapter }{UnderlyingToricVariety (for IsToricDivisor)}}
\logpage{[ 8, 4, 9 ]}\nobreak
\hyperdef{L}{X7ACCB998849BC939}{}
{\noindent\textcolor{FuncColor}{$\triangleright$\enspace\texttt{UnderlyingToricVariety({\mdseries\slshape divi})\index{UnderlyingToricVariety@\texttt{UnderlyingToricVariety}!for IsToricDivisor}
\label{UnderlyingToricVariety:for IsToricDivisor}
}\hfill{\scriptsize (attribute)}}\\
\textbf{\indent Returns:\ }
a variety 



 Returns the closure of the torus orbit corresponding to the prime divisor \mbox{\texttt{\mdseries\slshape divi}}. Not implemented for other divisors. Maybe we should add the support here. Is
this even a toric variety? Exercise left to the reader. }

 

\subsection{\textcolor{Chapter }{DegreeOfDivisor (for IsToricDivisor)}}
\logpage{[ 8, 4, 10 ]}\nobreak
\hyperdef{L}{X796D5B8B807A74B2}{}
{\noindent\textcolor{FuncColor}{$\triangleright$\enspace\texttt{DegreeOfDivisor({\mdseries\slshape divi})\index{DegreeOfDivisor@\texttt{DegreeOfDivisor}!for IsToricDivisor}
\label{DegreeOfDivisor:for IsToricDivisor}
}\hfill{\scriptsize (attribute)}}\\
\textbf{\indent Returns:\ }
an integer 



 Returns the degree of the divisor \mbox{\texttt{\mdseries\slshape divi}}. This is not to be confused with the (divisor) class of \mbox{\texttt{\mdseries\slshape divi}}! }

 

\subsection{\textcolor{Chapter }{VarietyOfDivisorpolytope (for IsToricDivisor)}}
\logpage{[ 8, 4, 11 ]}\nobreak
\hyperdef{L}{X7AE216928066B842}{}
{\noindent\textcolor{FuncColor}{$\triangleright$\enspace\texttt{VarietyOfDivisorpolytope({\mdseries\slshape divi})\index{VarietyOfDivisorpolytope@\texttt{VarietyOfDivisorpolytope}!for IsToricDivisor}
\label{VarietyOfDivisorpolytope:for IsToricDivisor}
}\hfill{\scriptsize (attribute)}}\\
\textbf{\indent Returns:\ }
a variety 



 Returns the variety corresponding to the polytope of the divisor \mbox{\texttt{\mdseries\slshape divi}}. }

 

\subsection{\textcolor{Chapter }{MonomsOfCoxRingOfDegree (for IsToricDivisor)}}
\logpage{[ 8, 4, 12 ]}\nobreak
\hyperdef{L}{X86C6946780CC6D38}{}
{\noindent\textcolor{FuncColor}{$\triangleright$\enspace\texttt{MonomsOfCoxRingOfDegree({\mdseries\slshape divi})\index{MonomsOfCoxRingOfDegree@\texttt{MonomsOfCoxRingOfDegree}!for IsToricDivisor}
\label{MonomsOfCoxRingOfDegree:for IsToricDivisor}
}\hfill{\scriptsize (attribute)}}\\
\textbf{\indent Returns:\ }
a list 



 Returns the monoms in the Cox ring of degree equal to the (divisor) class of
the divisor \mbox{\texttt{\mdseries\slshape divi}}. }

 

\subsection{\textcolor{Chapter }{CoxRingOfTargetOfDivisorMorphism (for IsToricDivisor)}}
\logpage{[ 8, 4, 13 ]}\nobreak
\hyperdef{L}{X874157CD83352D1C}{}
{\noindent\textcolor{FuncColor}{$\triangleright$\enspace\texttt{CoxRingOfTargetOfDivisorMorphism({\mdseries\slshape divi})\index{CoxRingOfTargetOfDivisorMorphism@\texttt{CoxRingOfTargetOfDivisorMorphism}!for IsToricDivisor}
\label{CoxRingOfTargetOfDivisorMorphism:for IsToricDivisor}
}\hfill{\scriptsize (attribute)}}\\
\textbf{\indent Returns:\ }
a ring 



 A basepoint free divisor \mbox{\texttt{\mdseries\slshape divi}} defines a map from its ambient variety in a projective space. This method
returns the Cox ring of such a projective space. }

 

\subsection{\textcolor{Chapter }{RingMorphismOfDivisor (for IsToricDivisor)}}
\logpage{[ 8, 4, 14 ]}\nobreak
\hyperdef{L}{X861A15E57E618D7F}{}
{\noindent\textcolor{FuncColor}{$\triangleright$\enspace\texttt{RingMorphismOfDivisor({\mdseries\slshape divi})\index{RingMorphismOfDivisor@\texttt{RingMorphismOfDivisor}!for IsToricDivisor}
\label{RingMorphismOfDivisor:for IsToricDivisor}
}\hfill{\scriptsize (attribute)}}\\
\textbf{\indent Returns:\ }
a ring map 



 A basepoint free divisor \mbox{\texttt{\mdseries\slshape divi}} defines a map from its ambient variety in a projective space. This method
returns the morphism between the cox ring of this projective space to the cox
ring of the ambient variety of \mbox{\texttt{\mdseries\slshape divi}}. }

 }

 
\section{\textcolor{Chapter }{Methods}}\label{Chapter_Toric_divisors_Section_Methods}
\logpage{[ 8, 5, 0 ]}
\hyperdef{L}{X8606FDCE878850EF}{}
{
  

\subsection{\textcolor{Chapter }{VeryAmpleMultiple (for IsToricDivisor)}}
\logpage{[ 8, 5, 1 ]}\nobreak
\hyperdef{L}{X7CE4802383A152FF}{}
{\noindent\textcolor{FuncColor}{$\triangleright$\enspace\texttt{VeryAmpleMultiple({\mdseries\slshape divi})\index{VeryAmpleMultiple@\texttt{VeryAmpleMultiple}!for IsToricDivisor}
\label{VeryAmpleMultiple:for IsToricDivisor}
}\hfill{\scriptsize (operation)}}\\
\textbf{\indent Returns:\ }
a divisor 



 Returns a very ample multiple of the ample divisor \mbox{\texttt{\mdseries\slshape divi}}. The method will fail if divisor is not ample. }

 

\subsection{\textcolor{Chapter }{CharactersForClosedEmbedding (for IsToricDivisor)}}
\logpage{[ 8, 5, 2 ]}\nobreak
\hyperdef{L}{X833D071081208C60}{}
{\noindent\textcolor{FuncColor}{$\triangleright$\enspace\texttt{CharactersForClosedEmbedding({\mdseries\slshape divi})\index{CharactersForClosedEmbedding@\texttt{CharactersForClosedEmbedding}!for IsToricDivisor}
\label{CharactersForClosedEmbedding:for IsToricDivisor}
}\hfill{\scriptsize (operation)}}\\
\textbf{\indent Returns:\ }
a list 



 Returns characters for closed embedding defined via the ample divisor \mbox{\texttt{\mdseries\slshape divi}}. The method fails if the divisor \mbox{\texttt{\mdseries\slshape divi}} is not ample. }

 

\subsection{\textcolor{Chapter }{\texttt{\symbol{92}}+ (for IsToricDivisor, IsToricDivisor)}}
\logpage{[ 8, 5, 3 ]}\nobreak
\hyperdef{L}{X834DD02A859B8363}{}
{\noindent\textcolor{FuncColor}{$\triangleright$\enspace\texttt{\texttt{\symbol{92}}+({\mdseries\slshape divi1, divi2})\index{\texttt{\symbol{92}}+@\texttt{\texttt{\symbol{92}}+}!for IsToricDivisor, IsToricDivisor}
\label{bSlash+:for IsToricDivisor, IsToricDivisor}
}\hfill{\scriptsize (operation)}}\\
\textbf{\indent Returns:\ }
a divisor 



 Returns the sum of the divisors \mbox{\texttt{\mdseries\slshape divi1}} and \mbox{\texttt{\mdseries\slshape divi2}}. }

 

\subsection{\textcolor{Chapter }{\texttt{\symbol{92}}- (for IsToricDivisor, IsToricDivisor)}}
\logpage{[ 8, 5, 4 ]}\nobreak
\hyperdef{L}{X7F80A78D7B0DB930}{}
{\noindent\textcolor{FuncColor}{$\triangleright$\enspace\texttt{\texttt{\symbol{92}}-({\mdseries\slshape divi1, divi2})\index{\texttt{\symbol{92}}-@\texttt{\texttt{\symbol{92}}-}!for IsToricDivisor, IsToricDivisor}
\label{bSlash-:for IsToricDivisor, IsToricDivisor}
}\hfill{\scriptsize (operation)}}\\
\textbf{\indent Returns:\ }
a divisor 



 Returns the divisor \mbox{\texttt{\mdseries\slshape divi1}} minus \mbox{\texttt{\mdseries\slshape divi2}}. }

 

\subsection{\textcolor{Chapter }{\texttt{\symbol{92}}* (for IsInt, IsToricDivisor)}}
\logpage{[ 8, 5, 5 ]}\nobreak
\hyperdef{L}{X859E5FF3794129CB}{}
{\noindent\textcolor{FuncColor}{$\triangleright$\enspace\texttt{\texttt{\symbol{92}}*({\mdseries\slshape k, divi})\index{\texttt{\symbol{92}}*@\texttt{\texttt{\symbol{92}}*}!for IsInt, IsToricDivisor}
\label{bSlash*:for IsInt, IsToricDivisor}
}\hfill{\scriptsize (operation)}}\\
\textbf{\indent Returns:\ }
a divisor 



 Returns \mbox{\texttt{\mdseries\slshape k}} times the divisor \mbox{\texttt{\mdseries\slshape divi}}. }

 

\subsection{\textcolor{Chapter }{MonomsOfCoxRingOfDegree (for IsToricVariety, IsHomalgElement)}}
\logpage{[ 8, 5, 6 ]}\nobreak
\hyperdef{L}{X80B40FD58011D036}{}
{\noindent\textcolor{FuncColor}{$\triangleright$\enspace\texttt{MonomsOfCoxRingOfDegree({\mdseries\slshape vari, elem})\index{MonomsOfCoxRingOfDegree@\texttt{MonomsOfCoxRingOfDegree}!for IsToricVariety, IsHomalgElement}
\label{MonomsOfCoxRingOfDegree:for IsToricVariety, IsHomalgElement}
}\hfill{\scriptsize (operation)}}\\
\textbf{\indent Returns:\ }
a list 



 Returns the monoms of the Cox ring of the variety \mbox{\texttt{\mdseries\slshape vari}} with degree equal to the class group element \mbox{\texttt{\mdseries\slshape elem}}. The variable \mbox{\texttt{\mdseries\slshape elem}} can also be a list. }

 

\subsection{\textcolor{Chapter }{DivisorOfGivenClass (for IsToricVariety, IsHomalgElement)}}
\logpage{[ 8, 5, 7 ]}\nobreak
\hyperdef{L}{X789DC1A47E8D22D9}{}
{\noindent\textcolor{FuncColor}{$\triangleright$\enspace\texttt{DivisorOfGivenClass({\mdseries\slshape vari, elem})\index{DivisorOfGivenClass@\texttt{DivisorOfGivenClass}!for IsToricVariety, IsHomalgElement}
\label{DivisorOfGivenClass:for IsToricVariety, IsHomalgElement}
}\hfill{\scriptsize (operation)}}\\
\textbf{\indent Returns:\ }
a divisor 



 Computes a divisor of the variety \mbox{\texttt{\mdseries\slshape divi}} which is member of the divisor class presented by \mbox{\texttt{\mdseries\slshape elem}}. The variable \mbox{\texttt{\mdseries\slshape elem}} can be a homalg element or a list presenting an element. }

 

\subsection{\textcolor{Chapter }{AddDivisorToItsAmbientVariety (for IsToricDivisor)}}
\logpage{[ 8, 5, 8 ]}\nobreak
\hyperdef{L}{X7CEE2AF37AABA035}{}
{\noindent\textcolor{FuncColor}{$\triangleright$\enspace\texttt{AddDivisorToItsAmbientVariety({\mdseries\slshape divi})\index{AddDivisorToItsAmbientVariety@\texttt{AddDivisorToItsAmbientVariety}!for IsToricDivisor}
\label{AddDivisorToItsAmbientVariety:for IsToricDivisor}
}\hfill{\scriptsize (operation)}}\\


 Adds the divisor \mbox{\texttt{\mdseries\slshape divi}} to the Weil divisor list of its ambient variety. }

 

\subsection{\textcolor{Chapter }{Polytope (for IsToricDivisor)}}
\logpage{[ 8, 5, 9 ]}\nobreak
\hyperdef{L}{X821ED50B856BDF79}{}
{\noindent\textcolor{FuncColor}{$\triangleright$\enspace\texttt{Polytope({\mdseries\slshape divi})\index{Polytope@\texttt{Polytope}!for IsToricDivisor}
\label{Polytope:for IsToricDivisor}
}\hfill{\scriptsize (operation)}}\\
\textbf{\indent Returns:\ }
a polytope 



 Returns the polytope of the divisor \mbox{\texttt{\mdseries\slshape divi}}. Another name for \emph{PolytopeOfDivisor} for compatibility and shortness. }

 

\subsection{\textcolor{Chapter }{CoxRingOfTargetOfDivisorMorphism (for IsToricDivisor, IsString)}}
\logpage{[ 8, 5, 10 ]}\nobreak
\hyperdef{L}{X84EDC7BC7BB19FBE}{}
{\noindent\textcolor{FuncColor}{$\triangleright$\enspace\texttt{CoxRingOfTargetOfDivisorMorphism({\mdseries\slshape divi, string})\index{CoxRingOfTargetOfDivisorMorphism@\texttt{CoxRingOfTargetOfDivisorMorphism}!for IsToricDivisor, IsString}
\label{CoxRingOfTargetOfDivisorMorphism:for IsToricDivisor, IsString}
}\hfill{\scriptsize (operation)}}\\
\textbf{\indent Returns:\ }
a ring 



 Given a toric divisor \mbox{\texttt{\mdseries\slshape divi}}, it induces a toric morphism. The target of this morphism is a toric variety.
This method returns the Cox ring of this target. The variables are named
according to \mbox{\texttt{\mdseries\slshape string}}. }

 }

 
\section{\textcolor{Chapter }{Constructors}}\label{Chapter_Toric_divisors_Section_Constructors}
\logpage{[ 8, 6, 0 ]}
\hyperdef{L}{X86EC0F0A78ECBC10}{}
{
  

\subsection{\textcolor{Chapter }{DivisorOfCharacter (for IsHomalgElement, IsToricVariety)}}
\logpage{[ 8, 6, 1 ]}\nobreak
\hyperdef{L}{X84823A8C80B00518}{}
{\noindent\textcolor{FuncColor}{$\triangleright$\enspace\texttt{DivisorOfCharacter({\mdseries\slshape elem, vari})\index{DivisorOfCharacter@\texttt{DivisorOfCharacter}!for IsHomalgElement, IsToricVariety}
\label{DivisorOfCharacter:for IsHomalgElement, IsToricVariety}
}\hfill{\scriptsize (operation)}}\\
\textbf{\indent Returns:\ }
a divisor 



 Returns the divisor of the toric variety \mbox{\texttt{\mdseries\slshape vari}} which corresponds to the character \mbox{\texttt{\mdseries\slshape elem}}. }

 

\subsection{\textcolor{Chapter }{DivisorOfCharacter (for IsList, IsToricVariety)}}
\logpage{[ 8, 6, 2 ]}\nobreak
\hyperdef{L}{X814A40377CB29DC8}{}
{\noindent\textcolor{FuncColor}{$\triangleright$\enspace\texttt{DivisorOfCharacter({\mdseries\slshape lis, vari})\index{DivisorOfCharacter@\texttt{DivisorOfCharacter}!for IsList, IsToricVariety}
\label{DivisorOfCharacter:for IsList, IsToricVariety}
}\hfill{\scriptsize (operation)}}\\
\textbf{\indent Returns:\ }
a divisor 



 Returns the divisor of the toric variety \mbox{\texttt{\mdseries\slshape vari}} which corresponds to the character which is created by the list \mbox{\texttt{\mdseries\slshape lis}}. }

 

\subsection{\textcolor{Chapter }{CreateDivisor (for IsHomalgElement, IsToricVariety)}}
\logpage{[ 8, 6, 3 ]}\nobreak
\hyperdef{L}{X78D779C681E9C1D8}{}
{\noindent\textcolor{FuncColor}{$\triangleright$\enspace\texttt{CreateDivisor({\mdseries\slshape elem, vari})\index{CreateDivisor@\texttt{CreateDivisor}!for IsHomalgElement, IsToricVariety}
\label{CreateDivisor:for IsHomalgElement, IsToricVariety}
}\hfill{\scriptsize (operation)}}\\
\textbf{\indent Returns:\ }
a divisor 



 Returns the divisor of the toric variety \mbox{\texttt{\mdseries\slshape vari}} which corresponds to the Weil group element \mbox{\texttt{\mdseries\slshape elem}}. by the list \mbox{\texttt{\mdseries\slshape lis}}. }

 

\subsection{\textcolor{Chapter }{CreateDivisor (for IsList, IsToricVariety)}}
\logpage{[ 8, 6, 4 ]}\nobreak
\hyperdef{L}{X7836CA92784E98B0}{}
{\noindent\textcolor{FuncColor}{$\triangleright$\enspace\texttt{CreateDivisor({\mdseries\slshape lis, vari})\index{CreateDivisor@\texttt{CreateDivisor}!for IsList, IsToricVariety}
\label{CreateDivisor:for IsList, IsToricVariety}
}\hfill{\scriptsize (operation)}}\\
\textbf{\indent Returns:\ }
a divisor 



 Returns the divisor of the toric variety \mbox{\texttt{\mdseries\slshape vari}} which corresponds to the Weil group element which is created by the list \mbox{\texttt{\mdseries\slshape lis}}. }

 }

 }

   
\chapter{\textcolor{Chapter }{Blowups of toric varieties}}\label{Chapter_Blowups_of_toric_varieties}
\logpage{[ 9, 0, 0 ]}
\hyperdef{L}{X7BABCA3984A9F55C}{}
{
  
\section{\textcolor{Chapter }{Constructors}}\label{Chapter_Blowups_of_toric_varieties_Section_Constructors}
\logpage{[ 9, 1, 0 ]}
\hyperdef{L}{X86EC0F0A78ECBC10}{}
{
  

\subsection{\textcolor{Chapter }{BlowupOfToricVariety (for IsToricVariety, IsList, IsString)}}
\logpage{[ 9, 1, 1 ]}\nobreak
\hyperdef{L}{X87EFC9AA7DF9981C}{}
{\noindent\textcolor{FuncColor}{$\triangleright$\enspace\texttt{BlowupOfToricVariety({\mdseries\slshape a, toric, variety, a, list, and, a, string})\index{BlowupOfToricVariety@\texttt{BlowupOfToricVariety}!for IsToricVariety, IsList, IsString}
\label{BlowupOfToricVariety:for IsToricVariety, IsList, IsString}
}\hfill{\scriptsize (operation)}}\\
\textbf{\indent Returns:\ }
a variety 



 The arguments are a toric variety vari, a string s which specifies the locus
to be blown up and a string which specifies how to name the new blowup
coordinate. Based on this, this method creates the blowup of a toric variety.
This process rests on a star sub-division of the fan (c.f. 3.3.17 in
Cox-Little-Schenk) }

 

\subsection{\textcolor{Chapter }{SequenceOfBlowupsOfToricVariety (for IsToricVariety, IsList)}}
\logpage{[ 9, 1, 2 ]}\nobreak
\hyperdef{L}{X81A6EDC085349C26}{}
{\noindent\textcolor{FuncColor}{$\triangleright$\enspace\texttt{SequenceOfBlowupsOfToricVariety({\mdseries\slshape a, toric, variety, and, a, list})\index{SequenceOfBlowupsOfToricVariety@\texttt{SequenceOfBlowupsOfToricVariety}!for IsToricVariety, IsList}
\label{SequenceOfBlowupsOfToricVariety:for IsToricVariety, IsList}
}\hfill{\scriptsize (operation)}}\\
\textbf{\indent Returns:\ }
a variety 



 The arguments are a toric variety vari and a list of lists. Each entry of this
list must contain the information for one blowup, i.e. be made up of the two
lists used as input for the method BlowupOfToricVariety. This method then
performs this sequence of blowups and returns the corresponding toric variety. }

 }

 }

 \def\indexname{Index\logpage{[ "Ind", 0, 0 ]}
\hyperdef{L}{X83A0356F839C696F}{}
}

\cleardoublepage
\phantomsection
\addcontentsline{toc}{chapter}{Index}


\printindex

\immediate\write\pagenrlog{["Ind", 0, 0], \arabic{page},}
\newpage
\immediate\write\pagenrlog{["End"], \arabic{page}];}
\immediate\closeout\pagenrlog
\end{document}
