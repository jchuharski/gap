
%%%%%%%%%%%%%%%%%%%%%%%%%%%%%%%%%%%%%%%%%%%%%%%%%%%%%%%%%%%%%%%%%%%%%%%%%%%%%
\Chapter{Nilpotent Quotients}

This chapter contains a description of the nilpotent quotient algorithm
for associative finitely presented algebras. We refer to \cite{Eic11} for 
background on the algorithms used in this Chapter.

%%%%%%%%%%%%%%%%%%%%%%%%%%%%%%%%%%%%%%%%%%%%%%%%%%%%%%%%%%%%%%%%%%%%%%%%%%%%%
\Section{Computing nilpotent quotients}

Let $A$ be a finitely presented algebra in the GAP sense. The following
function can be used to determine the class-$c$ nilpotent quotient of $A$.
The quotient is described by a nilpotent table.

\> NilpotentQuotientOfFpAlgebra( A, c ) F

The output of this function is a nilpotent table with some additional
entries. In particular, there is the additional entry $img$ which 
describes the images of the generators of $A$ in the nilpotent table.

%%%%%%%%%%%%%%%%%%%%%%%%%%%%%%%%%%%%%%%%%%%%%%%%%%%%%%%%%%%%%%%%%%%%%%%%%%%%%
\Section{Example of nilpotent quotient computation}

\beginexample
gap> F := FreeAssociativeAlgebra(GF(2), 2);;
gap> g := GeneratorsOfAlgebra(F);;
gap> r := [g[1]^2, g[2]^2];;
gap> A := F/r;;
gap> NilpotentQuotientOfFpAlgebra(A,3);
rec( def := [ 1, 2 ], dim := 8, fld := GF(2), 
  img := [ <a GF2 vector of length 8>, <a GF2 vector of length 8> ], 
  mat := [ [  ], [  ] ], rnk := 2, 
  tab := 
    [ [<a GF2 vector of length 8>, <a GF2 vector of length 8>, 
        [ 0*Z(2), 0*Z(2), 0*Z(2), 0*Z(2), Z(2)^0, 0*Z(2), 0*Z(2), 0*Z(2) ], 
        [ 0*Z(2), 0*Z(2), 0*Z(2), 0*Z(2), 0*Z(2), 0*Z(2), 0*Z(2), 0*Z(2) ], 
        [ 0*Z(2), 0*Z(2), 0*Z(2), 0*Z(2), 0*Z(2), 0*Z(2), 0*Z(2), 0*Z(2) ], 
        [ 0*Z(2), 0*Z(2), 0*Z(2), 0*Z(2), 0*Z(2), 0*Z(2), 0*Z(2), Z(2)^0 ], 
        [ 0*Z(2), 0*Z(2), 0*Z(2), 0*Z(2), 0*Z(2), 0*Z(2), 0*Z(2), 0*Z(2) ], 
        [ 0*Z(2), 0*Z(2), 0*Z(2), 0*Z(2), 0*Z(2), 0*Z(2), 0*Z(2), 0*Z(2) ] ], 
      [ <a GF2 vector of length 8>, <a GF2 vector of length 8>, 
        [ 0*Z(2), 0*Z(2), 0*Z(2), 0*Z(2), 0*Z(2), 0*Z(2), 0*Z(2), 0*Z(2) ], 
        [ 0*Z(2), 0*Z(2), 0*Z(2), 0*Z(2), 0*Z(2), Z(2)^0, 0*Z(2), 0*Z(2) ], 
        [ 0*Z(2), 0*Z(2), 0*Z(2), 0*Z(2), 0*Z(2), 0*Z(2), Z(2)^0, 0*Z(2) ], 
        [ 0*Z(2), 0*Z(2), 0*Z(2), 0*Z(2), 0*Z(2), 0*Z(2), 0*Z(2), 0*Z(2) ], 
        [ 0*Z(2), 0*Z(2), 0*Z(2), 0*Z(2), 0*Z(2), 0*Z(2), 0*Z(2), 0*Z(2) ], 
        [ 0*Z(2), 0*Z(2), 0*Z(2), 0*Z(2), 0*Z(2), 0*Z(2), 0*Z(2), 0*Z(2) ] ], 
      [ [ 0*Z(2), 0*Z(2), 0*Z(2), 0*Z(2), 0*Z(2), 0*Z(2), 0*Z(2), 0*Z(2) ], 
        [ 0*Z(2), 0*Z(2), 0*Z(2), 0*Z(2), 0*Z(2), Z(2)^0, 0*Z(2), 0*Z(2) ] ], 
      [ [ 0*Z(2), 0*Z(2), 0*Z(2), 0*Z(2), Z(2)^0, 0*Z(2), 0*Z(2), 0*Z(2) ], 
        [ 0*Z(2), 0*Z(2), 0*Z(2), 0*Z(2), 0*Z(2), 0*Z(2), 0*Z(2), 0*Z(2) ] ]], 
  wds := [ ,, [ 2, 1 ], [ 1, 2 ], [ 1, 3 ], [ 2, 4 ], [ 2, 5 ], [ 1, 6 ] ], 
  wgs := [ 1, 1, 2, 2, 3, 3, 4, 4 ] )
\endexample
